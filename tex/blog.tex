\section{Choosing Between $\Z_p$ and $\Z_p^*$ in Cryptographic Schemes}
\label{sec:zp-vs-zpstar}

In pairing-based cryptographic protocols, the choice of algebraic structures for randomness and messages—specifically whether to use the full field $\Z_p$ (including 0) or its multiplicative subgroup $\Z_p^*$ (excluding 0)—has subtle but critical implications for security and correctness. Below, we analyze their tradeoffs.

\subsection{Why $\Z_p^*$ is Safer}
Using $\Z_p^*$ (non-zero exponents) is often preferred for \textit{blinding factors} and \textit{randomness} due to:
\begin{itemize}
    \item \textbf{Avoiding Degenerate Cases}: 
        A zero exponent (e.g., $r = 0$) can produce trivial group elements:
        \[
        g^0 = 1 \in \G_1, \quad 0 \cdot P = \mathcal{O} \in \text{EC groups},
        \]
        which may leak secrets or break protocol unlinkability. For instance, in PS-style commitments $\cm = g^t \prod Y_i^{m_i}$, a $t = 0$ would expose $\prod Y_i^{m_i}$.
    
    \item \textbf{Security Proof Compatibility}: 
        Many zero-knowledge proofs (e.g., Schnorr-type responses $z = \tilde{x} + c \cdot x$) implicitly assume $x \neq 0$ to avoid division-by-zero errors in reductions.
    
    \item \textbf{Invertibility Guarantees}: 
        Non-zero elements in $\Z_p^*$ are invertible, simplifying operations like computing $r^{-1} \mod p$ in signature schemes.
\end{itemize}

\subsection{Why Some Schemes Use $\Z_p$}
Despite the risks, protocols like the Pointcheval-Sanders (PS) scheme often sample from $\Z_p$ because:
\begin{itemize}
    \item \textbf{Pairing Algebraic Requirements}: 
        Pairing equations (e.g., $e(g^a, h^b) = e(g, h)^{ab}$) require exponents to span the full field $\Z_p$ to preserve algebraic relationships. For example, polynomial evaluations over $\Z_p$ (common in PS credentials) would fail if restricted to $\Z_p^*$.
    
    \item \textbf{Negligible Failure Probability}: 
        For large $p$, the probability of sampling $r = 0$ is $1/p$, which is considered cryptographically negligible. Schemes often accept this risk to simplify implementations.
    
    \item \textbf{Message Flexibility}: 
        Messages (e.g., attributes $m_j$) may need to include 0 as a valid value. For example, a credential might encode $m_j = 0$ to represent "no value" for an optional field.
\end{itemize}

\subsection{Practical Recommendations}
\begin{itemize}
    \item \textbf{For Blinding Factors}: Always use $\Z_p^*$ to eliminate edge cases and align with security assumptions in proofs.
    \item \textbf{For Messages/Attributes}: Use $\Z_p$ if 0 is a valid semantic value (e.g., default states).
    \item \textbf{In Pairing-Based Schemes}: Follow the scheme's specification—PS uses $\Z_p$ for exponents to maintain pairing correctness, but ensure other safeguards (e.g., range proofs) mitigate risks.
\end{itemize}

\paragraph{Summary} While $\Z_p^*$ is theoretically safer for randomness, practical schemes like PS often use $\Z_p$ for compatibility with pairing algebra. Developers must weigh algebraic requirements against edge-case risks when choosing structures.