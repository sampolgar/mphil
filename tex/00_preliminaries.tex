% 
% Literature Review https://arxiv.org/pdf/2501.07209
% 

\newpage
\section{Preliminaries}

\begin{definition}[Signature Scheme]
A signature scheme $\mathsf{Sig}$ is a tuple of probabilistic polynomial-time algorithms $(\mathsf{KeyGen}, \mathsf{Sign}, \mathsf{Verify})$ where:

\begin{itemize}
    \item $\mathsf{KeyGen}(1^\lambda) \rightarrow (\mathsf{sk}, \mathsf{pk})$: is a probabilistic algorithm that takes as input a security parameter $\lambda$ and outputs a secret signing key $\mathsf{sk}$ and a public verification key $\mathsf{pk}$. The message space $\mathcal{M}$ is implicitly defined by $\mathsf{pk}$.
    
    \item $\mathsf{Sign}(\mathsf{sk}, m; r) \rightarrow \sigma$: is a probabilistic algorithm that takes as input the secret key $\mathsf{sk}$, a message $m \in \mathcal{M}$, and random coins $r$ sampled from the randomness space $\mathcal{R}$. It outputs a signature $\sigma$.
    
    \item $\mathsf{Verify}(\mathsf{pk}, m, \sigma) \rightarrow b$: is a deterministic algorithm that takes as input the public key $\mathsf{pk}$, a message $m \in \mathcal{M}$, and a signature $\sigma$. It outputs a bit $b \in \{0,1\}$, where 1 indicates acceptance and 0 indicates rejection.
\end{itemize}

\end{definition}

\begin{definition}[Correctness]
A signature scheme $(\mathsf{KeyGen}, \mathsf{Sign}, \mathsf{Verify})$ is correct if for all $k \in \mathbb{N}$, all key pairs $(\mathsf{sk}, \mathsf{pk}) \in [\mathsf{KeyGen}(1^k)]$ and all $m \in \mathcal{M}$ we have:

$$\Pr[\mathsf{Verify}(m, \mathsf{Sign}(m, \mathsf{sk}), \mathsf{pk}) = 1] = 1.$$
\end{definition}

\begin{definition}[EUF-CMA Security]
A signature scheme $(\mathsf{KeyGen}, \mathsf{Sign}, \mathsf{Verify})$ is existentially unforgeable under adaptive chosen-message attacks if for all PPT algorithms $\mathcal{A}$, there exists a negligible function $\negl$ such that:
$$\Pr\left[\begin{array}{l}
    (\mathsf{sk}, \mathsf{pk}) \sample \mathsf{KeyGen}(1^\lambda) \\
    (m^*, \sigma^*) \sample \mathcal{A}^{\mathcal{O}_{\mathsf{sk}}}(\mathsf{pk})
\end{array} : \begin{array}{l}
    m^* \notin Q \land \\
    \mathsf{Verify}(m^*, \sigma^*, \mathsf{pk}) = 1
\end{array}\right] \leq \negl$$
where $Q$ is the set of queries made to $\mathcal{O}_{\mathsf{sk}}$ with access to $\mathsf{sk}$ is defined by:
\[
\text{Oracle }\mathcal{O}_{\mathsf{sk}}(m): \text{ Returns } \sigma \gets \mathsf{Sign}(m, \mathsf{sk})
\]
\end{definition}

\begin{definition}[Commitment Scheme]\label{def:commitmentscheme}
A commitment scheme $\mathsf{Com}$ is a tuple of probabilistic polynomial-time algorithms $(\mathsf{Setup}, \mathsf{Commit}, \mathsf{Open})$ where:
\begin{itemize}
    \item $\mathsf{Setup}(1^\lambda) \rightarrow \mathsf{ck}$: is a probabilistic algorithm that takes as input a security parameter $\lambda$ and outputs a commitment key $\mathsf{ck}$. The message space $\mathcal{M}$ is implicitly defined by $\mathsf{ck}$.
    
    \item $\mathsf{Commit}(\mathsf{ck}, m) \rightarrow (\mathsf{cm}, r)$: is a probabilistic algorithm that takes as input the commitment key $\mathsf{ck}$ and a message $m \in \mathcal{M}$. It outputs a commitment $\mathsf{cm}$ and an opening value $r$.
    
    \item $\mathsf{Open}(\mathsf{ck}, \mathsf{cm}, m, r) \rightarrow b$: is a deterministic algorithm that takes as input the commitment key $\mathsf{ck}$, a commitment $\mathsf{cm}$, a message $m$, and an opening value $r$. It outputs a bit $b \in \{0,1\}$, where 1 indicates a valid opening and 0 indicates an invalid opening.
\end{itemize}
\end{definition}

\begin{definition}[Correctness]
A commitment scheme $(\mathsf{Setup}, \mathsf{Commit}, \mathsf{Open})$ is correct if for all $\lambda \in \mathbb{N}$, all commitment keys $\mathsf{ck} \in [\mathsf{Setup}(1^\lambda)]$, and all messages $m \in \mathcal{M}$:
$$\Pr[(\mathsf{cm}, r) \sample \mathsf{Commit}(\mathsf{ck}, m) : \mathsf{Open}(\mathsf{ck}, \mathsf{cm}, m, r) = 1] = 1.$$
\end{definition}

\begin{definition}[Hiding]
A commitment scheme $(\mathsf{Setup}, \mathsf{Commit}, \mathsf{Open})$ is hiding if for all PPT adversaries $\mathcal{A}$, there exists a negligible function $\negl$ such that:
$$\left|\Pr\left[\begin{array}{l}
    \mathsf{ck} \sample \mathsf{Setup}(1^\lambda) \\
    (m_0, m_1) \sample \mathcal{A}(\mathsf{ck}) \\
    b \sample \{0,1\} \\
    (\mathsf{cm}, r) \sample \mathsf{Commit}(\mathsf{ck}, m_b) \\
    b' \sample \mathcal{A}(\mathsf{ck}, \mathsf{cm})
\end{array} : b' = b\right] - \frac{1}{2}\right| \leq \negl(\lambda)$$
\end{definition}

\begin{definition}[Binding]
A commitment scheme $(\mathsf{Setup}, \mathsf{Commit}, \mathsf{Open})$ is binding if for all PPT adversaries $\mathcal{A}$, there exists a negligible function $\negl$ such that:
$$\Pr\left[\begin{array}{l}
    \mathsf{ck} \sample \mathsf{Setup}(1^\lambda) \\
    (\mathsf{cm}, m_0, m_1, r_0, r_1) \sample \mathcal{A}(\mathsf{ck})
\end{array} : \begin{array}{l}
    m_0 \neq m_1 \land \\
    \mathsf{Open}(\mathsf{ck}, \mathsf{cm}, m_0, r_0) = 1 \land \\
    \mathsf{Open}(\mathsf{ck}, \mathsf{cm}, m_1, r_1) = 1
\end{array}\right] \leq \negl(\lambda)$$
\end{definition}

\subsection{Assumptions}
\begin{definition}[LRSW Assumption in Bilinear Groups]
For any PPT adversary $\mathcal{A}$, we say the LRSW assumption holds if there exists a negligible function $\negl$ such that:
$$\Pr\left[\begin{array}{l}
    \BG = (p, \G_1, \G_2, \G_T, e, g, \tilde{g}) \sample \BGGen(1^\lambda) \\
    x,y \sample \Z_p, X \gets g^x, \tilde{Y} \gets \tilde{g}^y \\
    (m^*, A, B, C) \sample \mathcal{A}^{\mathcal{O}_{x,y}}(\BG, X, \tilde{Y})
\end{array} : \begin{array}{l}
    m^* \notin Q \quad \land \quad \exists r \in \Z_p :\\
    A = g^r \land B = A^y \land C = A^{x+m^*xy}
\end{array}\right] \leq \negl(\lambda)$$
where $Q$ is the set of queries made to $\mathcal{O}_{x,y}$ with access to $x,y$ is defined by:
\[
\text{Oracle }\mathcal{O}_{x,y}(m): \text{ Samples } r \sample \Z_p, \text{ Returns } (g^r, (g^r)^y, (g^r)^{x+mxy})
\]
\end{definition}


\begin{definition}[PS-LRSW Assumption in Bilinear Groups]
For any PPT adversary $\mathcal{A}$, we say the PS-LRSW assumption holds if there exists a negligible function $\negl$ such that:
$$\Pr\left[\begin{array}{l}
    \BG = (p, \G_1, \G_2, \G_T, e, g, \tilde{g}) \sample \BGGen(1^\lambda) \\
    x,y \sample \Z_p, X \gets g^x, Y \gets g^y \\
    \tilde{X} \gets \tilde{g}^x, \tilde{Y} \gets \tilde{g}^y \\
    (m^*, P_1, P_2) \sample \mathcal{A}^{\mathcal{O}_{x,y}}(\BG, X, Y, \tilde{X}, \tilde{Y})
\end{array} : \begin{array}{l}
    m^* \notin Q \land P_1 \neq 1_{\G_1} \land \\
    e(P_1, \tilde{X} \cdot \tilde{Y}^{m^*}) = e(P_2, \tilde{g})
\end{array}\right] \leq \negl$$
where $Q$ is the set of queries made to $\mathcal{O}_{x,y}$ with access to $x,y$ is defined by:
\[
\text{Oracle }\mathcal{O}_{x,y}(m): \text{ Samples } h \sample \G_1, \text{ Returns the pair }P = (h, h^{x+my})
\]
\end{definition}



\begin{definition}[Type-3 PS-LRSW Assumption]
For any PPT adversary $\mathcal{A}$, we say the Type-3 PS-LRSW assumption holds in the generic bilinear group model if there exists a negligible function $\negl$ such that:
$$\Pr\left[\begin{array}{l}
    \BG = (p, \G_1, \G_2, \G_T, e, g, \tilde{g}) \sample \BGGen(1^\lambda) \\
    x,y \sample \Z_p, X \gets g^x, \tilde{Y} \gets \tilde{g}^y \\
    (m^*, P_1, P_2) \sample \mathcal{A}^{\mathcal{O}_{x,y}}(\BG, X, \tilde{Y})
\end{array} : \begin{array}{l}
    m^* \notin Q \land P_1 \neq 1_{\G_1} \land \\
    e(P_1, \tilde{X} \cdot \tilde{Y}^{m^*}) = e(P_2, \tilde{g})
\end{array}\right] \leq \negl$$
where $Q$ is the set of queries made to $\mathcal{O}_{x,y}$ with access to $x,y$ is defined by:
\[
\text{Oracle }\mathcal{O}_{x,y}(m): \text{ Samples } h \sample \G_1, \text{ Returns the pair } P = (h, h^{x+my})
\]

\noindent Furthermore, in the generic group model after $q$ oracle queries and $q_G$ group-oracle queries, the probability of success is bounded by $\mathcal{O}((q + q_G)^2/p)$.
\end{definition}



\begin{definition}[XDDH/SXDH Assumption in Bilinear Groups]
For any PPT adversary $\mathcal{A}$, we say the XDDH assumption holds in $\G_i$ ($i \in \{1,2\}$) if there exists a negligible function $\negl$ such that:
$$\left|\Pr\left[\begin{array}{l}
    \BG = (p, \G_1, \G_2, \G_T, e, g, \tilde{g}) \sample \BGGen(1^\lambda) \\
    a, b \sample \Z_p, c \gets ab \bmod p \\
    d \sample \mathcal{A}(\BG, g_i^a, g_i^b, g_i^c)
\end{array}\right] - 
\Pr\left[\begin{array}{l}
    \BG = (p, \G_1, \G_2, \G_T, e, g, \tilde{g}) \sample \BGGen(1^\lambda) \\
    a, b, r \sample \Z_p \\
    d \sample \mathcal{A}(\BG, g_i^a, g_i^b, g_i^r)
\end{array}\right]\right| \leq \negl$$
where $g_i$ is the generator of $\G_i$. The SXDH assumption holds if XDDH holds in both $\G_1$ and $\G_2$.
\end{definition}




\begin{definition}[Symmetric Discrete Logarithm Assumption (SDLP)]\label{sdlp}
For any PPT adversary $\mathcal{A}$, we say the SDLP assumption holds if there exists a negligible function $\negl$ such that:
$$\Pr\left[\begin{array}{l}
    \BG = (p, \G_1, \G_2, \G_T, e, g, \tilde{g}) \sample \BGGen(1^\lambda) \\
    x \sample \Z_p \\
    x' \sample \mathcal{A}(\BG, g^x, \tilde{g}^x)
\end{array} : x = x'\right] \leq \negl$$
where validity of input can be verified by checking $e(g, \tilde{g}^x) = e(g^x, \tilde{g})$.
\end{definition}



% Assumptions from here - as discussed by UTT paper commitment instantiation
% The binding property holds under the SDL assumption of \cite{hutchison_get_2010}
% which follows from 1-SDH of \cite{boneh_short_2008}. (See Section I for details.) 








% % % % % % % % % % 
% 
% END PRELIMINARIES
% 
% % % % % % % % % % 


























% % % % % % % % % % 
% 
% Randomizable Vector Commitments
% 
% % % % % % % % % % 


\newpage
\section{Pedersen Commitment Scheme for a vector of messages}\label{sec:commitment}
In this section, we introduce a specialized extension of Pedersen commitments that supports vector messages, rerandomizability, and position binding. Our main contribution is a security proof in the Algebraic Group Model (AGM) that establishes position binding based on the Symmetric Discrete Logarithm Problem (SDLP).

\subsection{Extended Properties}

Building on the standard commitment scheme defined in Section \ref{def:commitmentscheme}, we extend it with the following properties:

\begin{definition}[Rerandomizability]
A commitment scheme $\mathsf{Com} = (\mathsf{Setup}, \mathsf{Commit}, \mathsf{Open})$ is rerandomizable if it includes an additional algorithm $\mathsf{Rerand}$ such that:

\begin{itemize}
\item $\mathsf{Rerand}(\mathsf{ck}, \mathsf{cm}, r_\Delta) \rightarrow (\mathsf{cm}', r')$: is a probabilistic algorithm that takes as input the commitment key $\mathsf{ck}$, a commitment $\mathsf{cm}$, and additional randomness $r_\Delta$. It outputs a new commitment $\mathsf{cm}'$ and updated opening value $r'$.
\end{itemize}

For all $\lambda \in \mathbb{N}$, all $\mathsf{ck} \in [\mathsf{Setup}(1^\lambda)]$, all $m \in \mathcal{M}$, all $(\mathsf{cm}, r) \in [\mathsf{Commit}(\mathsf{ck}, m)]$, and all $r_\Delta$:
$$\Pr[(\mathsf{cm}', r') \sample \mathsf{Rerand}(\mathsf{ck}, \mathsf{cm}, r_\Delta) : \mathsf{Open}(\mathsf{ck}, \mathsf{cm}', m, r') = 1] = 1.$$

Furthermore, the distribution of $\mathsf{cm}'$ should be computationally indistinguishable from a fresh commitment to the same message.
\end{definition}


\begin{definition}[Position Binding]
A commitment scheme with message vectors in $\mathcal{M}^n$ is position binding if for all PPT adversaries $\mathcal{A}$, there exists a negligible function $\negl$ such that:
\[
    \Pr
    \left[
        \begin{array}{l}
        \mathsf{ck} \sample \mathsf{Setup}(1^\lambda, 1^n) \\
        (\mathsf{cm}, i, \vec{m}_0, \vec{m}_1, r_0, r_1) \sample \mathcal{A}(\mathsf{ck}) 
        \end{array}
        : \begin{array}{l}
            \mathsf{Open}(\mathsf{ck}, \mathsf{cm}, \vec{m}_0, r_0) = 1 \land \\
            \mathsf{Open}(\mathsf{ck}, \mathsf{cm}, \vec{m}_1, r_1) = 1 \land \\
            \vec{m}_0[i] \neq \vec{m}_1[i] \land \\
            \vec{m}_0[j] = \vec{m}_1[j] \; \forall j \neq i
          \end{array}
    \right] \leq \negl(\lambda)
\]
This ensures that an adversary cannot open a commitment to two different values at any single position while keeping other positions constant.
\end{definition}

\subsection{Dual-Group Construction}

We instantiate a rerandomizable vector commitment scheme in the bilinear group setting as per the construction in \cite{tomescu2022utt} to enable efficient integration with our signature scheme. Let $\G_1, \G_2$ be cyclic groups of prime order $p$ with an efficient Type-3 pairing $e: \G_1 \times \G_2 \to \G_T$.

\begin{itemize}
    \item $\mathsf{Setup}(1^\lambda, 1^n) \to \mathsf{ck}$:
    Sample generators $(g, \tilde{g}) \sample \G_1 \times \G_2$.
    For $i \in [1,n]$: Sample $y_i \sample \Z_p$, compute $(g_i, \tilde{g}_i) \gets (g^{y_i}, \tilde{g}^{y_i})$.
    Return $\mathsf{ck} \gets (g, (g_1,\ldots,g_n), \tilde{g}, (\tilde{g}_1,\ldots,\tilde{g}_n))$.
    
    \item $\mathsf{Commit}(\mathsf{ck}, \vec{m}) \to (\mathsf{cm}, \widetilde{\mathsf{cm}}, r)$:
    Parse $\mathsf{ck}$ as $(g, (g_1,\ldots,g_n), \tilde{g}, (\tilde{g}_1,\ldots,\tilde{g}_n))$.
    Sample $r \sample \Z_p$.
    Compute $\mathsf{cm} \gets g^r \prod_{i=1}^n g_i^{m_i}$ and $\widetilde{\mathsf{cm}} \gets \tilde{g}^r \prod_{i=1}^n \tilde{g}_i^{m_i}$.
    Return $(\mathsf{cm}, \widetilde{\mathsf{cm}}, r)$.
    
    \item $\mathsf{Open}(\mathsf{ck}, (\mathsf{cm}, \widetilde{\mathsf{cm}}), \vec{m}, r) \to \{0,1\}$:
    Check if $\mathsf{cm} = g^r \prod_{i=1}^n g_i^{m_i}$ and $\widetilde{\mathsf{cm}} = \tilde{g}^r \prod_{i=1}^n \tilde{g}_i^{m_i}$.
    Additionally, verify $e(\mathsf{cm}, \tilde{g}) = e(g, \widetilde{\mathsf{cm}})$.
    Return 1 if all checks pass, 0 otherwise.
    
    \item $\mathsf{Rerand}(\mathsf{ck}, (\mathsf{cm}, \widetilde{\mathsf{cm}}), \Delta_r) \to ({\mathsf{cm}}', \widetilde{\mathsf{cm}}', r')$:
    Compute ${\mathsf{cm}}' \gets \mathsf{cm} \cdot g^{\Delta_r}$ and $\widetilde{\mathsf{cm}}' \gets \widetilde{\mathsf{cm}} \cdot \tilde{g}^{\Delta_r}$.
    Set $r' \gets r + \Delta_r$.
    Return $({\mathsf{cm}}', \widetilde{\mathsf{cm}}', r')$.
\end{itemize}

This construction preserves the message vector while updating the randomness, making rerandomized commitments computationally indistinguishable from fresh commitments to the same message.


\subsection{Security Analysis}

We now prove that our construction satisfies position binding in the Algebraic Group Model (AGM).

\begin{proof}[Sketch]
We construct a reduction algorithm $\mathcal{B}$ that uses an adversary $\mathcal{A}$ against position binding to solve the SDLP problem \ref{sdlp}. $\mathcal{B}$ receives an SDLP instance $(g^x, \tilde{g}^x)$ and embeds it at a random position $i^* \in [1,n]$ in the commitment key:

\begin{enumerate}
    \item $\mathcal{B}$ sets $(g_{i^*}, \tilde{g}_{i^*}) = (g^x, \tilde{g}^x)$
    \item For all $j \neq i^*$, $\mathcal{B}$ generates $(g_j, \tilde{g}_j) = (g^{y_j}, \tilde{g}^{y_j})$ with $y_j \sample \Z_p$
    \item When $\mathcal{A}$ outputs a position binding break $(\mathsf{cm}, i, \vec{m}_0, \vec{m}_1, r_0, r_1)$, if $i = i^*$, then $\mathcal{B}$ can solve for $x$
\end{enumerate}

Since $\mathcal{A}$ is algebraic, $\mathcal{B}$ knows the representation of $\mathsf{cm}$ in terms of the group generators. When $i = i^*$ and $\mathcal{A}$ successfully breaks position binding, we have:
\[
    g^{r_0}g_1^{m_{0,1}}\cdots g_{i^*}^{m_{0,i^*}}\cdots g_n^{m_{0,n}} = g^{r_1}g_1^{m_{1,1}}\cdots g_{i^*}^{m_{1,i^*}}\cdots g_n^{m_{1,n}}
\]

Since $m_{0,j} = m_{1,j}$ for all $j \neq i^*$ and $m_{0,i^*} \neq m_{1,i^*}$, after simplification we get:
\[
    g^{r_0} (g^x)^{m_{0,i^*}} = g^{r_1} (g^x)^{m_{1,i^*}}
\]

Solving for $x$ yields:
\[
    x = \frac{r_1 - r_0}{m_{0,i^*} - m_{1,i^*}} \mod p
\]

The reduction succeeds whenever $i = i^*$ (probability $1/n$) and $\mathcal{A}$ breaks position binding, giving the stated bound.
\end{proof}

The hiding property follows directly from the perfect hiding of Pedersen commitments, while binding follows from the discrete logarithm assumption. For completeness, this construction also satisfies the standard homomorphic properties of Pedersen commitments, enabling efficient aggregation and zero-knowledge proofs in our credential system.

A full proof can be found in the appendix \ref{appendix:commitmentreduction}
% % % % % % % % % % 
% 
% END Randomizable Vector Commitments
% 
% % % % % % % % % % 















% % % % % % % % % % 
% 
% PS Signature
% 
% % % % % % % % % % 
\newpage
\section{Rerandomizable Signatures over Commitments}\label{sec:pssignature}
We build upon the rerandomizable signature scheme over commitments introduced in UTT~\cite{tomescu2022utt}. This signature scheme enables our anonymous credential system to present signatures that are both unlinkable and unforgeable without revealing the underlying identity attributes. Our main contribution is a complete and tight security proof in the Algebraic Group Model that establishes $\EUFCMA$ security with minimal assumptions.


\begin{definition}[Rerandomizable Signature over Commitments]
A rerandomizable signature scheme over commitments $\mathsf{RS}$ extends a standard signature scheme with the following interface:
\begin{itemize}
    \item $\mathsf{KeyGen}(1^\lambda, \mathsf{ck}) \rightarrow (\mathsf{sk}, \mathsf{pk})$: Takes security parameter $\lambda$ and commitment key $\mathsf{ck}$, outputs signing key $\mathsf{sk}$ and verification key $\mathsf{pk}$.
    
    \item $\mathsf{Sign}(\mathsf{sk}, \mathsf{cm}; u) \rightarrow \sigma$: Signs commitment $\mathsf{cm}$ using signing key $\mathsf{sk}$ and randomness $u$.
    
    \item $\mathsf{Rerand}(\mathsf{pk}, \sigma, r_\Delta, u_\Delta) \rightarrow \sigma'$: Creates a rerandomized signature $\sigma'$ from signature $\sigma$ using randomization values $r_\Delta, u_\Delta$.
    
    \item $\mathsf{Verify}(\mathsf{pk}, \mathsf{cm}, \sigma) \rightarrow \{0,1\}$: Verifies signature $\sigma$ on commitment $\mathsf{cm}$ using public key $\mathsf{pk}$.
\end{itemize}
\end{definition}

\begin{definition}[Correctness]
A rerandomizable signature scheme over commitments satisfies correctness if for all security parameters $\lambda$, all key pairs $(\mathsf{sk}, \mathsf{pk}) \in [\mathsf{KeyGen}(1^\lambda, \mathsf{ck})]$, all messages $m \in \mathcal{M}$, all valid commitments $\mathsf{cm} = \mathsf{CM.Commit}(\mathsf{ck}, m, r)$, and all randomness values $u, r_\Delta, u_\Delta$:

\begin{enumerate}
    \item \textbf{Basic Verification}: $\mathsf{Verify}(\mathsf{pk}, \mathsf{cm}, \mathsf{Sign}(\mathsf{sk}, \mathsf{cm}; u)) = 1$
    
    \item \textbf{Rerandomization Consistency}: $\mathsf{Verify}(\mathsf{pk}, \mathsf{CM.Rerand}(\mathsf{ck}, \mathsf{cm}, r_\Delta), \mathsf{Rerand}(\mathsf{pk}, \sigma, r_\Delta, u_\Delta)) = 1$ where $\sigma = \mathsf{Sign}(\mathsf{sk}, \mathsf{cm}; u)$
    
    \item \textbf{Rerandomization Equivalence}: The distribution of $\mathsf{Rerand}(\mathsf{pk}, \mathsf{Sign}(\mathsf{sk}, \mathsf{cm}; u), r_\Delta, u_\Delta)$ is computationally indistinguishable from $\mathsf{Sign}(\mathsf{sk}, \mathsf{CM.Rerand}(\mathsf{ck}, \mathsf{cm}, r_\Delta); u+u_\Delta)$
\end{enumerate}
\end{definition}


\begin{definition}[EUF-CMA]
A rerandomizable signature scheme over commitments is existentially unforgeable under adaptive chosen message (commitment) attacks if for all PPT adversaries $\mathcal{A}$, there exists a negligible function $\negl$ such that:
    \begin{align*}
        &\Pr\left[
            \begin{array}{l}
                \mathsf{BG} \gets \mathsf{BGGen}(1^{\secparam}), \\
                \mathsf{ck} \gets \mathsf{CM.KeyGen}(\mathsf{BG}), \\
                (\sk, \pk) \gets \mathsf{KeyGen}(\mathsf{BG}), \\
                (m^*, \cm^*, \sigma^*) \gets \mathcal{A}^{\mathsf{Sign}(\sk, \cdot)}(\pk) \\
                \end{array}
                \quad : \quad
                \begin{array}{l}
                \cm^* = \mathsf{CM.Com}(\mathsf{ck}, m^*, r^*) \land \\
                \mathsf{RS.Ver}(\pk, \sigma^*, m^*) = 1 \land \\
                \cm^* \notin Q_{\cm}
            \end{array}
        \right] \leq \negl
    \end{align*}
where $Q_{\cm}$ is the set of all commitments queried to the signing oracle
\end{definition}




\subsubsection{Construction}\label{sig-construction}
We assume the existence of a commitment key $\ck$ from $\mathsf{CM.Setup}$ as input into our rerandomizable signature scheme $\mathsf{RS}$. We copy the algorithm below for convenience.
\begin{itemize}
    \item $\mathsf{CM.Setup}(\secparam, \ell, (y_i, \ldots, y_{\ell} \in \Z_p^{\ell})) \to \ck:$  
    Sample $(g, \tilde{g}) \sample \G_1 \times \G_2$, For $i \in [1,\ell]$: Compute $g_i = g^{y_i}$ and $\tilde{g}_i = \tilde{g}^{y_i}$. Return $\ck \gets (g, (g_1,\ldots,g_\ell), \tilde{g}, (\tilde{g}_1,\ldots,\tilde{g}_\ell))$
    
    \item $\mathsf{RS.KeyGen}(\secparam, \ck) \to (\sk, \vk):$ 
        Retrieve $(g, \cdot, \tilde{g}, \cdot)$ from $\mathsf{ck}$,
        Sample $x \sample \Z_p$,
        Set $(\sk, \vk) \gets (g^x, \tilde{g}^x)$, return $(\sk, \vk))$
        % Return $(\mathsf{sk} = (x,g), \mathsf{pk} = (\mathsf{pp}, \mathsf{vk}, \mathsf{ck}))$
    
    \item $\mathsf{RS.Sign}(\mathsf{sk}, \mathsf{cm}; u) \to \sigma:$ 
        Let $h \gets g^u$
        Return $\sigma \gets (h, (\sk \cdot \mathsf{cm})^u)$
    
    \item $\mathsf{RS.Rerand}(\sigma, r_\Delta, u_\Delta) \to \sigma':$
        Parse $\sigma$ as $(\sigma_1, \sigma_2)$
        Set $\sigma_1' \gets \sigma_1^{u_\Delta}$
        Set $\sigma_2' \gets (\sigma_2 \cdot \sigma_1^{r_\Delta})^{u_\Delta}$
        Return $\sigma' \gets (\sigma_1', \sigma_2')$
    
    \item $\mathsf{RS.Ver}(\vk, \cm, \sigma) \to \bit:$
        Parse $\sigma$ as $(\sigma_1, \sigma_2)$, The prover $\Prover$ runs a Proof of Knowledge protocol with the following relation 
    \[
        \mathcal{R} \gets \mathsf{PoK}\{(m_1,\ldots,m_\ell, r + r_\Delta): 
    \]
    \[
         e(\sigma_2', \tilde{g}) = e(\sigma_1', \vk)\cdot e(\sigma_1', \widetilde{\cm}') \quad \wedge \quad
        e(\cm', \tilde{g}) = e(g, \widetilde{\cm}') \quad \wedge \quad
        \cm' = g^{r + r_\Delta} \prod_{i=1}^\ell g_i^{m_i}
        \}
    \]

        \item $\mathsf{RS.VerKey}(\sk, \vk, \ck) \to \bit:$ verifies the correctness of the issuers secret and verification key $(\sk, \vk)$ and commitment key $\ck$:
        \[
        \mathcal{R}_{\mathsf{verkey}} = \{\pk = (\vk, \ck),(\sk, x, \{y_i\}_{i=1}^\ell) | sk = g^x \wedge vk = \tilde{g}^x \bigwedge_{i=1}^\ell (g_i = g^{y_i} \wedge \tilde{g}_i = \tilde{g}^{y_i})\}
        \]
        
\end{itemize}



\subsection{Security Analysis}

\begin{theorem}
The rerandomizable signature scheme is correct.
\end{theorem}


\begin{proof}
First we demonstrate the prover's rerandomized signature verifies with the verification key $\mathsf{vk}$ and the rerandomized commitment. Essentially, we need the following pairing to hold:
\[
    e(\sigma_2', \tilde{g}) = e(\sigma_1', \mathsf{vk} \cdot \widetilde{\mathsf{cm}'})
\]

We manipulate the bilinearity properties of the pairing groups to verify this equation:
    
\begin{align*}
    e(\sigma_2', \tilde{g}) &= e((\mathsf{sk} \cdot \mathsf{cm})^{u \cdot u_\Delta} \cdot h^{u_\Delta \cdot r_\Delta}, \tilde{g}) \\
    &= e(h^{x \cdot u_\Delta} \cdot \mathsf{cm}^{u \cdot u_\Delta} \cdot h^{u_\Delta \cdot r_\Delta}, \tilde{g}) \\
    &= e(h^{x \cdot u_\Delta}, \tilde{g}) \cdot e(\mathsf{cm}^{u \cdot u_\Delta}, \tilde{g}) \cdot e(h^{u_\Delta \cdot r_\Delta}, \tilde{g}) \\
    &= e(h^{u_\Delta}, \tilde{g}^x) \cdot e(\mathsf{cm}, \tilde{g})^{u \cdot u_\Delta} \cdot e(h^{u_\Delta}, \tilde{g})^{r_\Delta} \\
    &= e(\sigma_1', \mathsf{vk}) \cdot e(g^{u \cdot u_\Delta}, \widetilde{\mathsf{cm}}) \cdot e(\sigma_1', \tilde{g})^{r_\Delta} \\
    &= e(\sigma_1', \mathsf{vk}) \cdot e(\sigma_1', \widetilde{\mathsf{cm}}) \cdot e(\sigma_1', \tilde{g}^{r_\Delta}) \\
    &= e(\sigma_1', \mathsf{vk} \cdot \widetilde{\mathsf{cm}} \cdot \tilde{g}^{r_\Delta}) \\
    &= e(\sigma_1', \mathsf{vk} \cdot \widetilde{\mathsf{cm}'}) \\
\end{align*}
\end{proof}

Secondly, we need to verify knowledge of messages within the commitment. The Prover used $\widetilde{\mathsf{cm}'} \in \G_2$ during verification, which would be the natural method for a sigma-style proof of knowledge protocol, proving knowledge of the attributes of the commitment with $\G_2$ bases. However, due to the properties of the symmetric bilinear commitment \ref{sdlp}, we can prove the equality of $\mathsf{cm}' \in \G_1$ and $\widetilde{\mathsf{cm}'} \in \G_2$ to reduce $\G_2$ computation on both the prover and verifier during verification. 
Thus the prover computes the following to prove equality of commitments across groups:
\[
    e(\mathsf{cm}', \tilde{g}) = e(g, \widetilde{\mathsf{cm}}')
\]
Then proves the opening of the commitment in zero knowledge:
\[
 \pircom  \gets \mathsf{ZKPoK}\{(m_1,\ldots,m_\ell, r + r_\Delta): \cm' = g^{r + r_\Delta} \prod_{i=1}^\ell g_i^{m_i}
\]



\begin{theorem}
    RS is correct
\end{theorem}
\begin{proof}
    First we demonstrate the provers rerandomized signature verifies with the verification key $\vk$ and the rerandomized commitment. Essentially, we need the following pairing to hold
    \[
          e(\sigma_2', \tilde{g}) = e(\sigma_1', \vk \cdot \widetilde{\cm'})
    \]

    We manipulate the bilinearity properties of the pairing groups to verify the initial pairing.
    
    \begin{align*}
        e(\sigma_2', \tilde{g}) &= e((\sk \cdot \cm)^{u \cdot u\Delta}\cdot h^{u_\Delta \cdot r_\Delta}, \tilde{g}) \\
        &= e(h^{x \cdot u_\Delta}\cdot \cm^{u \cdot u_\Delta} \cdot h^{u_\Delta \cdot r_\Delta}, \tilde{g}) \\
        &= e(h^{x \cdot u_\Delta}, \tilde{g}) \cdot e(\cm^{u \cdot u_\Delta}, \tilde{g}) \cdot e(h^{ u_\Delta \cdot r_\Delta}, \tilde{g}) \\
        &= e(h^{u_\Delta}, \tilde{g}^x) \cdot e(\cm, \tilde{g})^{u \cdot u_\Delta} \cdot e(h^{ u_\Delta}, \tilde{g})^{r_\Delta} \\
        &= e(\sigma_1', \vk) \cdot e(g^{u \cdot u_\Delta}, \widetilde{\cm}) \cdot e(\sigma_1', \tilde{g})^{r_\Delta} \\
        &= e(\sigma_1', \vk) \cdot e(\sigma_1', \widetilde{\cm}) \cdot e(\sigma_1', \tilde{g}^{r_\Delta}) \\
        &= e(\sigma_1', \vk \cdot \widetilde{\cm} \cdot \tilde{g}^{r_\Delta}) \\
        &= e(\sigma_1', \vk \cdot \widetilde{\cm'}) \\
    \end{align*}

    Secondly, we need to verify knowledge of messages within the commitment. The Prover used $\widetilde{\cm'} \in \G_2$ during verification and this would be the natural method to for a sigma style proof of knowledge protocol, proving knowledge of the attributes of the commitment with $\G_2$ bases. However, due to the properties of the symmetric bilinear commitment \ref{sdlp}, we can prove the equality of $\cm' \in \G_1$ and $\widetilde{\cm'} \in \G_2$ to reduce $\G_2$ computation on both the prover and verifier during verification. 
    Thus the prover computes 
    \[
        e(\cm', \tilde{g}) = e(g, \widetilde{\cm}')
    \]
    Then runs a sigma protocol to prove
    \[
    \cm' = g^{r \cdot r_\Delta} \prod_{i=1}^\ell g_i^{m_i}
    \]

\end{proof}




\begin{theorem}[EUF-CMA Security]
Assume the PS-LRSW assumption holds and the Pedersen commitment is computationally binding. Then, in the Algebraic Group Model, our rerandomizable signature scheme is existentially unforgeable under adaptive chosen-message(commitment) attacks. For any algebraic PPT adversary $\mathcal{A}$, there exist PPT reductions $\mathcal{B}_0, \mathcal{B}_1$ such that:
\[
\Adv^{\mathsf{EUF\mbox{-}CMA}}_{\mathsf{RS},\mathcal{A}}(\lambda) \leq \Adv^{\mathsf{PS\mbox{-}LRSW}}_{\mathcal{B}_0}(\lambda) + \Adv^{\mathsf{Binding}}_{\mathcal{B}_1}(\lambda) + \frac{q_v + q_s}{p},
\]
where $q_v$ (verification) and $q_s$ (signing) are query counts.
\end{theorem}

\begin{proof}
We construct two reductions handling different forgery types. Let $\mathcal{A}$ be an adversary with advantage $\epsilon$.

\textbf{Reduction Strategy: } Since we don't know in advance which type of forgery $\AdvA$ will produce, we design two reductions:
\begin{itemize}
    \item $\mathcal{B}_0$ handles forgeries with a new message combination
    \item $\mathcal{B}_1$ handles forgeries where the same message appears in different commitments
\end{itemize}

\paragraph{1. Setup}
Given a PS-LRSW challenge $(g, \tilde{g}, X=g^x, \tilde{X}=\tilde{g}^x, Y=g^y, \tilde{Y}=\tilde{g}^y)$, both reductions:
\begin{itemize}
    \item Sample $\alpha_1, \alpha_2, \beta_1, \beta_2 \sample \Z_p$ 
    \item Set commitment base elements $g_i = Y^{\alpha_i}g^{\beta_i}$ and $\tilde{g}_i = \tilde{Y}^{\alpha_i}\tilde{g}^{\beta_i}$ for $i \in \{1,2\}$
    \item Set $\mathsf{pk} = (\tilde{X}, \mathsf{ck} = (g, g_1, g_2, \tilde{g}, \tilde{g}_1, \tilde{g}_2))$
    \item Send $\mathsf{pk}$ to $\mathcal{A}$
\end{itemize}

\textbf{Oracle Simulation:} 
For a signing query on commitment $\mathsf{cm} = g_1^{m_1}g_2^{m_2}g^r$:
\begin{itemize}
    \item $\mathcal{B}_0$ (PS-LRSW Reduction):
    \begin{enumerate}
        \item Compute $m = \alpha_1m_1 + \alpha_2m_2$ 
        \item Query PS-LRSW oracle to get $(h, h^{x+my})$
        \item Return $\sigma = (h, h^{x+my} \cdot h^{\beta_1m_1 + \beta_2m_2 + r})$
    \end{enumerate}
    
    \item $\mathcal{B}_1$ (Binding Reduction):
    \begin{enumerate}
        \item Sample $u \sample \Z_p$
        \item Compute $\sigma = (g^u, (X \cdot \mathsf{cm})^u)$
    \end{enumerate}
\end{itemize}

\noindent \textbf{Verification Oracle:}

\begin{itemize}
    \item Parse $\sigma = (\sigma_1, \sigma_2)$
    \item Check $e(\sigma_2, \tilde{g}) = e(\sigma_1, \tilde{X} \cdot \widetilde{\mathsf{cm}})$ where $\widetilde{\mathsf{cm}} = \tilde{g}_1^{m_1}\tilde{g}_2^{m_2}\tilde{g}^r$
    \item Use AGM to extract exponents from $\sigma_1, \sigma_2$ if needed
\end{itemize}


\textbf{Forgery Analysis:} 
When $\mathcal{A}$ outputs a forgery $(m_1^*, m_2^*, r^*, \sigma^* = (\sigma_1^*, \sigma_2^*))$:

\textbf{Case 1:} If $m^* = \alpha_1m_1^* + \alpha_2m_2^*$ was never queried:
\begin{itemize}
    \item Since $\mathcal{A}$ is algebraic, $\mathcal{B}_0$ knows the representation of $\sigma_1^*$ as $g^u$
    \item $\mathcal{B}_0$ can compute $(g^u, \sigma_2^* / (\sigma_1^*)^{\beta_1m_1^* + \beta_2m_2^* + r^*})$
    \item This equals $(g^u, (g^x \cdot g^{ym^*})^u)$, breaking PS-LRSW
\end{itemize}

\textbf{Case 2:} If $m^*$ was queried but with different message components:
\begin{itemize}
    \item There exists a previous query $(m_1, m_2) \neq (m_1^*, m_2^*)$ but $\alpha_1m_1 + \alpha_2m_2 = \alpha_1m_1^* + \alpha_2m_2^*$
    \item This gives $g_1^{m_1}g_2^{m_2} = g_1^{m_1^*}g_2^{m_2^*}$, breaking the binding property
    \item $\mathcal{B}_1$ can extract the discrete logarithm relations, solving the binding challenge
\end{itemize}
\end{proof}



\paragraph{Intuition for Security}
Our dual reduction approach handles all possible forgery types. The key insight is embedding the PS-LRSW challenge in a structured way that ensures:
\begin{enumerate}
    \item If the adversary forges a signature for a new linear combination of messages, we break PS-LRSW
    \item If the adversary reuses a linear combination but with different individual messages, we break binding
\end{enumerate}
We ensure a forgery must break one underlying assumption, the tight reduction only uses a factor related to the number of oracle queries. 


































\newpage
\section{Appendix}

\subsection{Pedersen Commitment Reduction}\label{appendix:commitmentreduction}
\begin{theorem}
    In the Algebraic Group Model, if the Symmetric Discrete Logarithm Problem (SDLP) is hard in the bilinear group $\BG$, then our Commitment scheme satisfies position binding. Specifically, for any algebraic PPT adversary $\mathcal{A}$ against position binding, there exists a PPT reduction $\mathcal{B}$ against SDLP such that:
    \[
        \mathsf{Adv}^{\mathsf{pos\text{-}bind}}_{\mathcal{A},\mathsf{RVC}}(\lambda) \leq \ell \cdot \mathsf{Adv}^{\mathsf{SDLP}}_{\mathcal{B},\mathbb{G}}(\lambda)
    \]
    where $\ell$ is the vector length.
\end{theorem}

\begin{proof}
We prove via reduction in the AGM. Given an algebraic PPT adversary $\mathcal{A}$ that breaks position binding with non-negligible probability $\epsilon$, we construct a PPT algorithm $\mathcal{B}$ that solves SDLP with probability $\epsilon/\ell$. For clarity, we illustrate with $\ell = 3$; the proof generalizes naturally.

Algorithm $\mathcal{B}$ works as follows:
\begin{enumerate}
    \item \textbf{Setup}: On input SDLP instance $(g^x, \tilde{g}^x) \in \mathbb{G}_1 \times \mathbb{G}_2$, $\mathcal{B}$ proceeds to:
    \begin{enumerate}
        \item Sample $i^* \sample [1,\ell]$ uniformly at random
        \item For position $i^*$: set $(g_{i^*}, \tilde{g}_{i^*}) \gets (g^x, \tilde{g}^x)$
        \item For positions $j \neq i^*$: sample $y_j \sample \mathbb{Z}_p$, set $(g_j, \tilde{g}_j) \gets (g^{y_j}, \tilde{g}^{y_j})$
        \item Give $\mathsf{ck} = ((g_1, g_2, g_3), (\tilde{g}_1, \tilde{g}_2, \tilde{g}_3))$ to $\mathcal{A}$
    \end{enumerate}
    
    \item \textbf{Position Binding Break}: Since $\mathcal{A}$ is algebraic, when it outputs $(\mathsf{cm}, i, \vec{m}_0, \vec{m}_1, r_0, r_1)$, it also provides the representation of $\mathsf{cm}$ in terms of the generators:
    \begin{itemize}
        \item $\mathsf{cm} \in \mathbb{G}_1$ with its algebraic representation
        \item $i \in [1,\ell]$ is the position where binding breaks
        \item $\vec{m}_0, \vec{m}_1 \in \mathbb{Z}_p^\ell$ differ only at position $i$
        \item $r_0, r_1 \in \mathbb{Z}_p$ are opening randomness values
    \end{itemize}
    
    \item \textbf{Extracting SDLP}: If $i \neq i^*$, abort. Otherwise:
    \begin{enumerate}
        \item By the algebraic property of $\mathcal{A}$, we have explicit representations of the commitment openings:
        \[
            g^{r_0}g_1^{m_{0,1}}g_2^{x \cdot m_{0,2}}g_3^{m_{0,3}} = g^{r_1}g_1^{m_{1,1}}g_2^{x \cdot m_{1,2}}g_3^{m_{1,3}}
        \]
        
        \item Since these representations are explicit in the AGM, we can directly compare exponents:
        \[
            r_0 + y_1m_{0,1} + xm_{0,2} + y_3m_{0,3} = r_1 + y_1m_{1,1} + xm_{1,2} + y_3m_{1,3}
        \]
        
        \item Since $\vec{m}_0$ and $\vec{m}_1$ differ only at position $i^*=2$, we have $m_{0,1}=m_{1,1}$ and $m_{0,3}=m_{1,3}$. Terms cancel:
        \[
            r_0 + xm_{0,2} = r_1 + xm_{1,2}
        \]
        
        \item Solve for $x$:
        \[
            x \equiv \frac{r_1-r_0}{m_{0,2}-m_{1,2}} \pmod{p}
        \]
        Note: Division is well-defined as $m_{0,2} \neq m_{1,2}$ by assumption.
    \end{enumerate}
\end{enumerate}

The reduction succeeds whenever $i = i^*$ and $\mathcal{A}$ succeeds, which occurs with probability $\epsilon/\ell$. This is non-negligible when $\epsilon$ is non-negligible, contradicting the SDLP assumption.
\end{proof}



We analyze the reduction's properties in detail:
\begin{itemize}
    \item \textbf{Perfect Simulation:} The commitment key distribution is identical to the real scheme:
        \begin{itemize}
            \item At position $i^*$: $(g_{i^*}, \tilde{g}_{i^*}) = (g^x, \tilde{g}^x)$ is uniformly distributed in $\G_1 \times \G_2$ by the SDLP instance properties
            \item At positions $j \neq i^*$: $(g_j, \tilde{g}_j) = (g^{y_j}, \tilde{g}^{y_j})$ is uniform due to $y_j \sample \Z_p$
            \item Therefore, from $\mathcal{A}$'s view, $\mathsf{ck}$ is distributed identically to the real scheme
        \end{itemize}
    
    \item \textbf{Extraction Success:} $\mathcal{B}$ successfully extracts the SDLP solution when:
        \begin{itemize}
            \item $\mathcal{A}$ outputs a valid position binding break (occurs with probability $\epsilon$)
            \item The guessed position matches: $i = i^*$ (occurs with probability $1/\ell$)
            \item The extraction equation is solvable: $m_{0,i^*} \neq m_{1,i^*}$ (guaranteed by definition of position binding break)
        \end{itemize}
    
    \item \textbf{Advantage Analysis:} Combining these probabilities:
        \begin{itemize}
            \item Events are independent as $i^*$ is chosen before $\mathcal{A}$'s execution
            \item $\mathsf{Pr}[\mathcal{B} \text{ succeeds}] = \epsilon \cdot \frac{1}{\ell}$
            \item Therefore: $\mathsf{Adv}^{\mathsf{SDLP}}_{\mathcal{B},\G}(\lambda) \geq \frac{1}{\ell} \cdot \mathsf{Adv}^{\mathsf{pos\text{-}bind}}_{\mathcal{A},\mathsf{RVC}}(\lambda)$
        \end{itemize}
\end{itemize}

Thus, if $\mathcal{A}$ breaks position binding with non-negligible probability $\epsilon$, then $\mathcal{B}$ solves SDLP with non-negligible probability $\epsilon/\ell$, contradicting the SDLP hardness assumption in $\G$.
