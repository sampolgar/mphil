\mychapter{Introduction}

\begin{enumerate}
    \item Chapter 1: Introduce the new Privacy Preserving Digital Identity landscape \\
    We need ABCs that can prove expressive statements efficiently, components that can be decentralized

    \item Chapter 2: Foundations: Optimized Expressive Predicate Proofs \\
    We need efficient, expressive proofs for credentials, so I optimized PS signatures and benchmarked it against other similar schemes

    \item Chapter 3: Multi-Issuer Multi-Credential System (MIMC-ABC) \\
    Using this, I solved the problem of securely combining credentials from multiple issuers and showed privacy has a small overhead.

    \item Chapter 4: Private Accountability: Sybil Resistance \\
    I solve the private accountability problem by adding a hierarchy, improving a nullifier scheme, and show it can be used for revocation

    \item Chapter 5: Threshold Sybil Resistant Identity System \\
    To make it even more robust, I thresholdized the scheme and show it's more efficient than sota in key use-cases

    \item Chapter 6: Conclusion and Future Work

    \item Appendices: Detailed Proofs, Sigma Protocols
    
\end{enumerate}



\newpage
\section{Motivation}
\subsubsection*{Overarching Research Problem: }
How can we build privacy-preserving credential systems that are simultaneously expressive, efficient, secure against malicious actors, resistant to abuse, and free from centralized trust?


\subsubsection*{Chapter 2: Foundations: } 
How can we construct anonymous credential systems that efficiently support expressive proofs—like range proofs, attribute equality, or set membership—while remaining secure against malicious issuers? Existing schemes either verify simple proofs (e.g., possession) efficiently (sps-eq, ACT) or handle complex predicates at high computational cost (zk-creds), often assuming honest issuers (ACT, Coconut). This chapter lays the foundation for a system that overcomes these limitations.

\noindent \textbf{Technical Challenges}
\begin{enumerate}
    \item Designing an Anonymous Credential scheme for efficient zero-knowledge proof of complex predicates without using zkSNARK
    \item Ensuring security for malicious issuers without affecting performance
    \item Balancing computational overhead with practical usability
\end{enumerate}

\subsubsection*{Chapter 3: Multi-Issuer Multi-Credential System: } 
How can users privately combine credentials from multiple, mutually distrusting issuers (e.g., government IDs and bank statements) to prove they belong to the same identity, especially in decentralized settings? Non-private systems easily verify credential consistency, but privacy-preserving approaches struggle to bind credentials securely without a trusted party, aggregate signatures \cite{mir_aggregate_2023} have trouble with revoking individual credentials from an aggregate.

\noindent \textbf{Technical Challenges}
\begin{enumerate}
    \item Defining and achieving identity binding across credentials without revealing the user’s identity.
    \item Preventing attacks where users mix credentials from different identities (e.g., credential swapping).
    \item Maintaining efficiency as the number of issuers and credentials scales.
\end{enumerate}



\subsubsection*{Chapter 4: Sybil Resistance: } 
How can we prevent Sybil attacks—where users create multiple identities to abuse services like voting or payments—in anonymous credential systems without compromising privacy? Traditional nullifier schemes either leak information or are too costly for multi-issuer scenarios.

\noindent \textbf{Technical Challenges}
\begin{enumerate}
    \item Creating unique, privacy-preserving bindings (e.g., nullifiers) for credentials without a central authority.
    \item Designing efficient nullifiers that scale to multi-issuer, multi-credential settings.
    \item Integrating these with zero-knowledge proofs to ensure verification doesn’t reveal identities.
\end{enumerate}



\subsubsection*{Chapter 5: Threshold Issuance: } 
How can we distribute trust in credential issuance to eliminate central points of failure, while preserving the efficiency and security of our anonymous credential system? Centralized issuers risk catastrophic breaches—malicious actors could issue unlimited credentials—yet adapting efficient schemes to a threshold model is complex, especially for multi-credential verification.

\noindent \textbf{Technical Challenges}
\begin{enumerate}
    \item Adapting our efficient signature scheme for threshold key generation and signing.
    \item Ensuring private, distributed issuance.
    \item Security against colluding or malicious threshold issuers
\end{enumerate}

\section{Contributions and Thesis Organization}

This thesis develops a comprehensive privacy-preserving digital identity framework that enables expressive credential verification, multi-issuer identity binding, and sybil resistance without compromising efficiency or security. My contributions span four interconnected areas:

\begin{enumerate}
    \item \textbf{Foundations: Expressive Predicates for Attribute-Based (Anonymous) Credentials (ABC's)} (Chapter 2)
    \begin{itemize}
        \item Extended rerandomizable signature scheme with formal position-binding security in the Algebraic Group Model
        \item Formalized security model for anonymity against colluding credential issuers
        \item Developed a G2-variant optimization reducing verification costs by [X\%]
        \item Demonstrated sub-linear practical performance using multi-scalar multiplication techniques
        \item Benchmarked predicate proofs against SOTA showing on average [X\%] improvement
        \item Provided the first comprehensive benchmarks across BBS+ and PS signature variants
    \end{itemize}

    \item \textbf{Multi-Issuer Multi-Credential System (MIMC-ABC)} (Chapter 3)
    \begin{itemize}
        \item Formalized security model for multi-issuer identity binding with position-binding commitments
        \item Proved unforgeability and anonymity even against colluding credential issuers
        \item Demonstrated 30\% performance improvement over comparable schemes for multi-credential verification
    \end{itemize}
    
    \item \textbf{Private Accountability: Sybil Resistance} (Chapter 4)
        \begin{itemize}
            \item Introduced the Credential Relationship Binding Nullifier (CRBN), a pairing-free construction for preventing credential reuse while maintaining privacy
            \item Developed a novel zero-knowledge proof of multiplicative inverse enabling efficient verification of nullifier correctness
            \item Achieved 33\% faster nullifier evaluation and 60\% faster verification compared to state-of-the-art approaches
            \item Formalized security properties (uniqueness, unlinkability, and double-spending prevention) with proofs under the q-Diffie-Hellman Inversion assumption
        \end{itemize}

    \item \textbf{Threshold Sybil Resistant Identity System} (Chapter 5)
        \begin{itemize}
            \item Designed a threshold issuance protocol eliminating central points of trust while preserving all MIMC-ABC security properties
            \item Demonstrated 2-3× performance improvements over comparable threshold anonymous credential systems
            \item Integrated the threshold construction with sybil resistance mechanisms from Chapter 4, creating the first fully decentralized system with both properties
            \item Provided comprehensive benchmarks across varying threshold parameters to guide real-world deployments
        \end{itemize}

        \item \textbf{Open Source Library}
        \begin{itemize}
            \item Anonymous Credential Comparison
            \item Developed a G2-variant optimization reducing verification costs by [X\%]
            \item Demonstrated sub-linear practical performance using multi-scalar multiplication techniques
        \end{itemize}
\end{enumerate}
Collectively, these contributions address the fundamental tension between privacy and accountability in digital identity systems. The techniques developed in this thesis enable the construction of credential systems that simultaneously provide strong privacy guarantees, protection against abuse, expressiveness for complex policy verification, and practical efficiency. The implementations and benchmarks demonstrate that these theoretical advances translate to concrete performance improvements suitable for real-world deployment.
