\newpage
\begin{table}[h]
\centering
\begin{tabular}{|l|c|c|}
\hline
\textbf{Operation} & \textbf{Prover} & \textbf{Verifier} \\
\hline
\multicolumn{3}{|l|}{\textbf{Commitment Equality Method}} \\
\hline
G1 exponentiations & 4 & 5 \\
G1 additions & 2 & 4 \\
Fp multiplications & 3 & 0 \\
Fp additions & 3 & 0 \\
\hline
\multicolumn{3}{|l|}{\textbf{G1 VRF Method}} \\
\hline
G1 exponentiations & 11 & 11 \\
G1 additions & 4 & 7 \\
Fp multiplications & 6 & 0 \\
Fp additions & 6 & 0 \\
\hline
\multicolumn{3}{|l|}{\textbf{Pairing + VRF Method}} \\
\hline
G1 exponentiations & 2 & 3 \\
G2 exponentiations & 2 & 3 \\
G1 additions & 1 & 2 \\
G2 additions & 1 & 2 \\
Fp multiplications & 12 & 0 \\
Fp additions & 8 & 0 \\
GT exponentiations & 8 & 5 \\
GT multiplications & 6 & 4 \\
Pairings & 4 & 4 \\
\hline
\end{tabular}
\caption{Comparison of computational operations between G1 and Pairing methods}
\end{table}

\begin{table}[h]
\centering
\begin{tabular}{|l|r|}
\hline
\textbf{Operation} & \textbf{Time} \\
\hline
Full Pairing & 1.6218 ms \\
Miller Loop & 0.6931 ms \\
Final Exponentiation & 0.9287 ms \\
G1 Mixed Addition (Affine + Jacobian) & 672 ns \\
G1 Point Doubling (2P) & 414 ns \\
G2 Mixed Addition (Affine + Jacobian) & 2143 ns \\
G2 Point Doubling (2P) & 1302 ns \\
\hline
Estimated G1 Scalar Mult (255-bit) & 191.59 $\mu$s \\
\emph{\small(255 doublings + ~128 additions)} & \\
Estimated G2 Scalar Mult (255-bit) & 606.01 $\mu$s \\
\emph{\small(255 doublings + ~128 additions)} & \\
\hline
\end{tabular}
\caption{Performance metrics for arkworks BLS12-381 implementation. Scalar multiplication estimates assume naive double-and-add implementation without optimizations.}
\label{tab:arkworks-performance}
\end{table}
\footnotetext{The G1 and G2 scalar multiplication estimates are derived using a naive double-and-add implementation analysis for 255-bit scalars. For a random scalar $k$, we assume approximately 255 doubling operations (one per bit) and 128 addition operations (corresponding to an expected Hamming weight of $\frac{255}{2}$ for a random scalar). The G1 estimate of 191.59$\mu$s is computed as $(255 \times 414\text{ns}) + (128 \times 672\text{ns})$ using the measured doubling and mixed addition timings. Similarly, the G2 estimate of 606.01$\mu$s is computed as $(255 \times 1302\text{ns}) + (128 \times 2143\text{ns})$. These estimates represent upper bounds as they do not account for common optimizations such as windowing methods, NAF (Non-Adjacent Form) representations, or parallel computation strategies.}



\newpage

\begin{construction}{Proof of Zero($C$)}{zeroproof}
    \vspace{1em}
    \textbf{Public Parameters:} $g_1, g_2, h_1\in \G$

    \textbf{Inputs:} $C$ such that $C = g_1^{m}g_2h^{r}$, $\Prover$ knows $m, r \in \Z_q$.
    \vspace{1em}
        \begin{enumerate}
        \item $\Prover$ samples $\alpha, \rho \sample [q-1]$ and sends $T \gets g_1^{\alpha}g_2h_1^{\rho}$ 

        \item $\Verifier$ sends challenge $c \sample [q-1]$

        \item $\Prover$ sends $s \gets \alpha + cm, u \gets \rho + cr$

        \item $\Verifier$ verifies that $g_1^sg_2^ch_1^u = C^cT$
    \end{enumerate}
\end{construction}
    

\begin{theorem}
    Construction~\ref{construct:zeroproof} is a $\Sigma$-protocol for the relation: 
    \[
    \mathcal{R} = \left\{ (C, g_1,g_2,h, q),(m,r) \; | \; C = g_1^{m}g_2h_1^{r}  \right\}
    \]    
\end{theorem}

\begin{proof}
    Folklore
\end{proof}

\begin{theorem}[Perfect Completeness]
    Construction \ref{construct:zeroproof} is a $\Sigma-$protocol for the relation $\mathcal{R}$ with perfect completeness: 
\end{theorem}
\begin{proof}
    We prove completeness by showing that for any $(C, g, h, q),(m,r) \in \mathcal{R}$, when both $\Prover$ and $\Verifier$ follow the protocol, $\Verifier$ accepts with $\Pr$ = 1.

    Let $x = (C,g_1,g_2,h,q)$ be common input and $w = (m, r)$ be $\Prover$'s private input. Consider an execution of the protocol where:
    \begin{enumerate}
        \item $\Prover$ samples $\alpha, \rho \sample [q-1]$ and sends $T \gets g_1^{\alpha}g_2h^{\rho}$
        \item $\Verifier$ sends challenge $c \sample [q-1]$
        \item $\Prover$ responds with $s \gets \alpha+cm, u \gets \rho+cr$
    \end{enumerate}
    Verification holds by 
    \begin{align}
        g_1^sg_2^ch_1^u &\stackrel{?}{=} C^cT \notag \\
        g_1^{\alpha + cm}g_2^ch^{\rho + cr} &\stackrel{?}{=} (g_1^{m}g_2h^r)^c g_1^{\alpha}g_2h^{\rho} \notag \\
        g_1^{\alpha + cm}g_2^ch^{\rho + cr} &= g_1^{\alpha + cm}g_2^c h^{\rho + cr} \notag \\
    \end{align}
    Thus, an honest verifier always accepts an honest prover's proof.
\end{proof}


\begin{theorem}[Soundness]
    Construction \ref{construct:zeroproof} is a $\Sigma-$protocol for the relation $\mathcal{R}$ with soundness: 
\end{theorem}
\begin{proof}
    We prove completeness by showing that for any $(C, g, h, q),(m,r) \in \mathcal{R}$, when both $\Prover$ and $\Verifier$ follow the protocol, $\Verifier$ accepts with $\Pr$ = 1.

    Let $x = (C,g_1,g_2,h,q)$ be common input and $w = (m, r)$ be $\Prover$'s private input. Consider an execution of the protocol where:
    \begin{enumerate}
        \item $\Prover$ samples $\alpha, \rho \sample [q-1]$ and sends $T \gets g_1^{\alpha}g_2h^{\rho}$
        \item $\Verifier$ sends challenge $c \sample [q-1]$
        \item $\Prover$ responds with $s \gets \alpha+cm, u \gets \rho+cr$
    \end{enumerate}
    Verification holds by 
    \begin{align}
        g_1^sg_2^ch_1^u &\stackrel{?}{=} C^cT \notag \\
        g_1^{\alpha + cm}g_2^ch^{\rho + cr} &\stackrel{?}{=} (g_1^{m}g_2h^r)^c g_1^{\alpha}g_2h^{\rho} \notag \\
        g_1^{\alpha + cm}g_2^ch^{\rho + cr} &= g_1^{\alpha + cm}g_2^c h^{\rho + cr} \notag \\
    \end{align}
    Thus, an honest verifier always accepts an honest prover's proof.
\end{proof}





\newpage
Commitment Scheme
