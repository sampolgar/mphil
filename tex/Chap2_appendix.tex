% 
% Literature Review https://arxiv.org/pdf/2501.07209
% 
\newpage
\section{Appendix A}

\subsection{Pedersen Commitment Reduction}\label{appendix:commitmentreduction}
\begin{theorem}
    In the Algebraic Group Model, if the Symmetric Discrete Logarithm Problem (SDLP) is hard in the bilinear group $\BG$, then our Commitment scheme satisfies position binding. Specifically, for any algebraic PPT adversary $\mathcal{A}$ against position binding, there exists a PPT reduction $\mathcal{B}$ against SDLP such that:
    \[
        \mathsf{Adv}^{\mathsf{pos\text{-}bind}}_{\mathcal{A},\mathsf{RVC}}(\lambda) \leq \ell \cdot \mathsf{Adv}^{\mathsf{SDLP}}_{\mathcal{B},\mathbb{G}}(\lambda)
    \]
    where $\ell$ is the vector length.
\end{theorem}

\begin{proof}
We prove via reduction in the AGM. Given an algebraic PPT adversary $\mathcal{A}$ that breaks position binding with non-negligible probability $\epsilon$, we construct a PPT algorithm $\mathcal{B}$ that solves SDLP with probability $\epsilon/\ell$. For clarity, we illustrate with $\ell = 3$; the proof generalizes naturally.

Algorithm $\mathcal{B}$ works as follows:
\begin{enumerate}
    \item \textbf{Setup}: On input SDLP instance $(g^x, \tilde{g}^x) \in \mathbb{G}_1 \times \mathbb{G}_2$, $\mathcal{B}$ proceeds to:
    \begin{enumerate}
        \item Sample $i^* \sample [1,\ell]$ uniformly at random
        \item For position $i^*$: set $(g_{i^*}, \tilde{g}_{i^*}) \gets (g^x, \tilde{g}^x)$
        \item For positions $j \neq i^*$: sample $y_j \sample \mathbb{Z}_p$, set $(g_j, \tilde{g}_j) \gets (g^{y_j}, \tilde{g}^{y_j})$
        \item Give $\mathsf{ck} = ((g_1, g_2, g_3), (\tilde{g}_1, \tilde{g}_2, \tilde{g}_3))$ to $\mathcal{A}$
    \end{enumerate}
    
    \item \textbf{Position Binding Break}: Since $\mathcal{A}$ is algebraic, when it outputs $(\mathsf{cm}, i, \vec{m}_0, \vec{m}_1, r_0, r_1)$, it also provides the representation of $\mathsf{cm}$ in terms of the generators:
    \begin{itemize}
        \item $\mathsf{cm} \in \mathbb{G}_1$ with its algebraic representation
        \item $i \in [1,\ell]$ is the position where binding breaks
        \item $\vec{m}_0, \vec{m}_1 \in \mathbb{Z}_p^\ell$ differ only at position $i$
        \item $r_0, r_1 \in \mathbb{Z}_p$ are opening randomness values
    \end{itemize}
    
    \item \textbf{Extracting SDLP}: If $i \neq i^*$, abort. Otherwise:
    \begin{enumerate}
        \item By the algebraic property of $\mathcal{A}$, we have explicit representations of the commitment openings:
        \[
            g^{r_0}g_1^{m_{0,1}}g_2^{x \cdot m_{0,2}}g_3^{m_{0,3}} = g^{r_1}g_1^{m_{1,1}}g_2^{x \cdot m_{1,2}}g_3^{m_{1,3}}
        \]
        
        \item Since these representations are explicit in the AGM, we can directly compare exponents:
        \[
            r_0 + y_1m_{0,1} + xm_{0,2} + y_3m_{0,3} = r_1 + y_1m_{1,1} + xm_{1,2} + y_3m_{1,3}
        \]
        
        \item Since $\vec{m}_0$ and $\vec{m}_1$ differ only at position $i^*=2$, we have $m_{0,1}=m_{1,1}$ and $m_{0,3}=m_{1,3}$. Terms cancel:
        \[
            r_0 + xm_{0,2} = r_1 + xm_{1,2}
        \]
        
        \item Solve for $x$:
        \[
            x \equiv \frac{r_1-r_0}{m_{0,2}-m_{1,2}} \pmod{p}
        \]
        Note: Division is well-defined as $m_{0,2} \neq m_{1,2}$ by assumption.
    \end{enumerate}
\end{enumerate}

The reduction succeeds whenever $i = i^*$ and $\mathcal{A}$ succeeds, which occurs with probability $\epsilon/\ell$. This is non-negligible when $\epsilon$ is non-negligible, contradicting the SDLP assumption.
\end{proof}



We analyze the reduction's properties in detail:
\begin{itemize}
    \item \textbf{Perfect Simulation:} The commitment key distribution is identical to the real scheme:
        \begin{itemize}
            \item At position $i^*$: $(g_{i^*}, \tilde{g}_{i^*}) = (g^x, \tilde{g}^x)$ is uniformly distributed in $\G_1 \times \G_2$ by the SDLP instance properties
            \item At positions $j \neq i^*$: $(g_j, \tilde{g}_j) = (g^{y_j}, \tilde{g}^{y_j})$ is uniform due to $y_j \sample \Z_p$
            \item Therefore, from $\mathcal{A}$'s view, $\mathsf{ck}$ is distributed identically to the real scheme
        \end{itemize}
    
    \item \textbf{Extraction Success:} $\mathcal{B}$ successfully extracts the SDLP solution when:
        \begin{itemize}
            \item $\mathcal{A}$ outputs a valid position binding break (occurs with probability $\epsilon$)
            \item The guessed position matches: $i = i^*$ (occurs with probability $1/\ell$)
            \item The extraction equation is solvable: $m_{0,i^*} \neq m_{1,i^*}$ (guaranteed by definition of position binding break)
        \end{itemize}
    
    \item \textbf{Advantage Analysis:} Combining these probabilities:
        \begin{itemize}
            \item Events are independent as $i^*$ is chosen before $\mathcal{A}$'s execution
            \item $\mathsf{Pr}[\mathcal{B} \text{ succeeds}] = \epsilon \cdot \frac{1}{\ell}$
            \item Therefore: $\mathsf{Adv}^{\mathsf{SDLP}}_{\mathcal{B},\G}(\lambda) \geq \frac{1}{\ell} \cdot \mathsf{Adv}^{\mathsf{pos\text{-}bind}}_{\mathcal{A},\mathsf{RVC}}(\lambda)$
        \end{itemize}
\end{itemize}

Thus, if $\mathcal{A}$ breaks position binding with non-negligible probability $\epsilon$, then $\mathcal{B}$ solves SDLP with non-negligible probability $\epsilon/\ell$, contradicting the SDLP hardness assumption in $\G$.


\section{Appendix B}\label{app:zkp}
\subsection{Zero-Knowledge Proofs and Sigma-Protocols}

\subsubsection{Interactive Proof Systems}
An interactive proof system for a language $L$ involves a prover $\mathcal{P}$ and verifier $\mathcal{V}$, both probabilistic polynomial-time machines. For a statement $x$ and witness $w$ where $x \in L$, it satisfies:
\begin{itemize}
    \item \textbf{Completeness}: $\Pr[(\mathcal{P}(w), \mathcal{V})(x) = 1] \geq 1 - \negl(\lambda)$.
    \item \textbf{Soundness}: For any $\mathcal{P}^*$, if $x \notin L$, $\Pr[(\mathcal{P}^*, \mathcal{V})(x) = 1] \leq \negl(\lambda)$.
\end{itemize}
In our system, $\mathcal{P}$ is the user proving credential validity, and $\mathcal{V}$ is the verifier.

\subsubsection{Zero-Knowledge Property}
A proof is zero-knowledge if it reveals only that $x \in L$. Formally, for any $\mathcal{V}^*$, there exists a simulator $\mathcal{S}$ such that:
\[
\{\text{VIEW}_{\mathcal{V}^*}(\mathcal{P}(w), \mathcal{V}^*)(x)\} \approx_c \{\mathcal{S}(x)\}, \quad \forall x \in L.
\]
This property ensures anonymity in our $\textsf{Show}$ protocol (Section~\ref{sec:anonymity}).

\subsubsection{Proofs of Knowledge}
A proof is a proof of knowledge if an extractor $\mathcal{E}$, with rewind access to $\mathcal{P}^*$, can extract $w$ when $\mathcal{P}^*$ convinces $\mathcal{V}$ with non-negligible probability:
\[
\Pr[\mathcal{E}^{\mathcal{P}^*}(x) = w : (x, w) \in R] \geq \Pr[(\mathcal{P}^*, \mathcal{V})(x) = 1] - \negl(\lambda).
\]
This supports unforgeability (Section~\ref{sec:unforgeability}), ensuring only users knowing $\textsf{usk}$ can generate valid proofs.

\subsubsection{Sigma-Protocols}\label{preliminaries-sigmaprotocol}
A Sigma-protocol for a relation $R$ is a three-move protocol:
\begin{enumerate}
    \item $\mathcal{P}(x, w)$ sends commitment $a$.
    \item $\mathcal{V}$ sends challenge $e \leftarrow \{0,1\}^t$.
    \item $\mathcal{P}$ sends response $z$.
\end{enumerate}
$\mathcal{V}$ accepts if $\phi(x, a, e, z) = 1$. It satisfies completeness, special soundness, and SHVZK.

\subsubsection{Example: Schnorr’s Protocol}
For $\mathcal{R}_{\mathsf{DL}} = \{(h, w) \in \mathbb{G} \times \mathbb{Z}_q : h = g^w\}$, Schnorr’s protocol proves knowledge of $w$:
\begin{itemize}
    \item \textbf{Commitment}: $\mathcal{P}$ samples $r \leftarrow \mathbb{Z}_q$, sends $a = g^r$.
    \item \textbf{Challenge}: $\mathcal{V}$ sends $e \leftarrow \{0,1\}^t$.
    \item \textbf{Response}: $\mathcal{P}$ sends $z = r + e w \mod q$.
    \item \textbf{Verification}: $\mathcal{V}$ checks $g^z = a \cdot h^e$.
\end{itemize}
\textbf{Properties}:
\begin{itemize}
    \item \textbf{Completeness}: $g^z = g^{r + e w} = g^r \cdot (g^w)^e = a \cdot h^e$.
    \item \textbf{Special Soundness}: From $(a, e, z)$ and $(a, e', z')$, $w = (z - z') / (e - e') \mod q$.
    \item \textbf{SHVZK}: Simulator samples $z \leftarrow \mathbb{Z}_q$, sets $a = g^z h^{-e}$.
\end{itemize}


% Add examples of schnorr protocols

% Appendix A: Sigma Protocol Constructions

% A.1 Notation and Conventions
%     - Define witness notation, challenge spaces, etc.

% A.2 Proof of Commitment Opening
%     - Complete protocol description: P→V, V→P, P→V
%     - Simulator construction
%     - Extractor construction

% A.3 Signature Verification Proof
%     - Protocol for proving e(σ₂′, g̃) = e(σ₁′, vk·c̃m′)
%     - Optimizations for pairing-based operations

% A.4 Predicate Evaluation Proofs
%     - Techniques for AND/OR compositions
%     - Range proofs for inequality predicates
%     - Equality proofs across credentials