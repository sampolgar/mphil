\chapter{Conclusion and Future Work}\label{chap7}
This thesis delivers a suite of fast, secure, and expressive tools for digital identity systems. By achieving Show+Verify times as low as 3.08 ms, securing against malicious issuers, and enabling multi-credential proofs in 72 ms, I’ve shown that the privacy-security-usability trilemma can be resolved. T-SIRIS and efficient nullifiers further enhance scalability and trust, while the Rust benchmarking library provides a foundation for future research.

\section{Summary}
This thesis advances the security–privacy–usability trilemma in digital identity systems through five core contributions:

\begin{itemize}
  \item \textbf{Chapter 2: Fast and Expressive Anonymous Credentials.} Introduced a new rerandomizable signature over commitments that reduces Show+Verify latency to 3.08 ms, secures against malicious issuers via mandatory key‐verification protocols, and demonstrates sublinear expressive proofs using $\Sigma$–protocol optimizations.
  
  \item \textbf{Chapter 3: Identity Binding for Multi‐Issuer, Multi‐Credential Proofs.} Formalized the Identity Binding property, enabling secure linkage of up to 16 heterogeneous credentials in a single proof at 72 ms, with optional aggregation reducing verification to 31 ms.
  
  \item \textbf{Chapter 4: Efficient Nullifiers from q‐DDHI.} Developed pairing‐free VRFs and three novel zero‐knowledge $\Sigma$-protocols to construct deterministic and rerandomizable nullifiers that achieve 5× speedups over prior work.
  
  \item \textbf{Chapter 5: T‐SIRIS — Threshold‐Issued, Sybil‐Resistant Identity System.} Designed and benchmarked a threshold issuance scheme that maintains near‐constant Show+Verify times across increasing issuers, outperforming state‐of‐the‐art threshold systems by up to 30x.
  
  \item \textbf{Chapter 6: Open‐Source Benchmark Library.} Delivered a Rust library with standardized microbenchmarks for anonymous credential schemes, revealing practical gains from multi‐scalar multiplication and optimized pairing routines.
\end{itemize}

\section{Future Work}
Key directions to extend this research include:
\begin{itemize}
  \item \emph{Post‐Quantum Instantiations:} Explore lattice‐based or hash‐based analogues of rerandomizable signatures and nullifiers to anticipate quantum threats.
  \item \emph{Rich Predicate Support:} Integrate circuit‐level SNARKs or bulletproofs to enable arbitrary boolean and threshold predicates without sacrificing sub‐10 ms verification.
  \item \emph{Real‐World Integration:} Prototype these primitives within production digital wallets (e.g., mobile SDKs, browser extensions) and evaluate end‐user latency, compatibility, and developer ergonomics.
  \item \emph{Dynamic Revocation and Auditing:} Design privacy‐preserving revocation mechanisms that support both immediate revocation checks and retrospective audits, balancing unlinkability with accountability.
  \item \emph{Adaptive Threshold Policies:} Investigate schemes where the issuance threshold adapts to context (e.g., higher thresholds for high‐value credentials) without re‐issuing keys.
  \item \emph{Hardware Acceleration:} Leverage secure enclaves or GPU/FPGA offloading to further reduce proof generation and verification costs on edge devices.
  \item \emph{Usability Studies:} Conduct human‐centered evaluations to quantify the impact of sub‐100 ms verifications on real‐world user experience and adoption metrics.
\end{itemize}

\clearpage