\mychapter{Credential Relationship Binding Nullifier}
Anonymous credential systems enable users to prove statements about their identity while maintaining privacy. In Chapters 2 and 3, we advanced from single-issuer Attribute-Based Anonymous Credentials ($\ABC$) to a Multi-Issuer Multi-Credential system ($\MIMCABC$) that securely binds credentials from different issuers to the same identity. However, practical identity frameworks require two additional capabilities: hierarchical credential organization and sybil resistance.

Real-world credentials naturally form a hierarchy, this is notable when applying for bank loans or new credentials from the government that require multiple credentials shown together, a naive request without privacy but challenging to implement in a privacy-preserving identity system. Consider a user with both a passport and a driver's license, as shown in Figure~\ref{fig:credential-nullifier}, the passport functions as a foundational Master Credential containing a secret key $\k$, while the driver's license is a Context Credential tied to a specific domain $\ctx$. For privacy and security, we need to cryptographically bind these credentials together and prevent users from obtaining multiple credentials for the same context (e.g., multiple driver's licenses) without compromising anonymity.

\begin{figure}
        \begin{pchstack}[boxed, center, space=4em]
            \begin{pcvstack}
                \procedure[space=auto]{Master Credential (Passport)}{%
                \id: 12345, \\
                \ctx: "master", \\
                \exp: "10/11/2026", \\
                \k: 54321
                }
            \end{pcvstack}
            \pcvspace
            \begin{pcvstack}
                \procedure[space=auto]{Context Credential (Driver's License)}{%
                 \id: 12345, \\
                 \ctx: "DMV", \\
                 \exp: "10/11/2028"
                }
            \end{pcvstack}
        \end{pchstack}
        \begin{center}
        \begin{tikzpicture}
            \draw[->, thick] (-2,0) -- (-1,-1);
            \draw[->, thick] (2,0) -- (1,-1);
        \end{tikzpicture}
        \end{center}
        \begin{pchstack}[boxed, center]
            \begin{pcvstack}
                \procedure[space=auto]{Nullifier}{%
                 \textsf{n} = 1/(\k + \text{"DMV"}) \\
                 \nul = g^{\textsf{n}} 
                }
            \end{pcvstack}
        \end{pchstack}
    \caption{A credential hierarchy with a nullifier that binds a context credential to a master credential. The nullifier combines the master credential's secret key $\k$ with the context identifier "DMV" through a one-way function, creating a unique, verifiable binding that prevents sybil attacks while maintaining privacy.}
    \label{fig:credential-nullifier}
\end{figure}

This presents a core technical challenge: \emph{How can we create a nullifier that cryptographically proves the relationship between two separate credentials without revealing the underlying attributes or enabling tracking across verifications?}

\section{SHOULD WE INCLUDE A CHAPTER ROADMAP?}

% \subsection*{Chapter Organization}
% The remainder of this chapter is organized as follows: Section 4.3 introduces our pairing-free VRF construction in prime-order groups. Section 4.4 presents our zero-knowledge proof protocol for multiplicative inverse relationships. Section 4.5 combines these components to construct the complete Credential Relationship Binding Nullifier (CRBN) system and demonstrates its integration with our identity framework. Finally, Section 4.6 provides a comprehensive performance evaluation comparing our approach to existing techniques.


\subsection{Technical Challenges and Approach}

Creating an efficient, privacy-preserving nullifier for credential binding raises three specific challenges:
\begin{enumerate}
    \item \textbf{Efficient Verification}: The nullifier must be verifiable without expensive bilinear pairings while maintaining provable uniqueness and verifiable pseudorandomness.
    
    \item \textbf{Privacy Preservation}: Verification must not reveal the master credential's key $\k$ or the context credential's identifier $\ctx$, nor enable credential linking.
    
    \item \textbf{Integration with $\MIMCABC$}: The nullifier mechanism must seamlessly extend our existing anonymous credential framework without compromising its security properties.
\end{enumerate}

Our approach leverages and extends Verifiable Random Functions (VRFs) to address these challenges. Specifically, we:

\begin{enumerate}
    \item Start with the Dodis-Yampolskiy VRF~\cite{hutchison_verifiable_2005}, which computes $y = e(g, \tilde{g})^{1/(sk+x)}$ in bilinear groups
    
    \item Develop a pairing-free variant that operates entirely in prime-order groups
    
    \item Create novel $\Sigma$-protocols for proving relationships between committed attributes and nullifiers and of independent interest
    
    \item Design a complete credential binding system that prevents sybil attacks while maintaining privacy
\end{enumerate}

\subsection{Related Work}

\begin{table}
\begin{center}
\caption{Comparison of Nullifier/VRF Schemes for Credential Binding}
\label{tab:nullifier-comparison}
\begin{tabular}{l|cccccc}
\toprule
\textbf{Scheme} & 
\textbf{Pairing-} & 
\textbf{Com. } & 
\textbf{Com. }  & 
\textbf{Deterministic} & 
\textbf{Proof} & 
\textbf{Relative} \\
 & 
 \textbf{Free} & 
 \textbf{Inputs}$^1$ & 
 \textbf{Outputs}$^2$ & 
 \textbf{Value}$^3$ & 
 \textbf{Type} & 
 \textbf{Ver. Time}$^4$ \\
\midrule
DY VRF \cite{hutchison_verifiable_2005} & \ding{55} & \ding{55} & \ding{55} & \ding{51} & Pairing & 3x slower \\
CanDID \cite{maram2021candid} & \ding{51} & \ding{51} & \ding{55} & \ding{51} & MPC & Very slow \\
TACT/S3ID \cite{rabaninejad_attribute-based_2024} & \ding{55} & \ding{51} & \ding{55} & \ding{51} & Groth Sahai + Pairing & ~1.5-2 slower \\
SyRA \cite{crites_syra_2024} & \ding{55} & \ding{51} & \ding{55} & \ding{51} & Sigma+Pairing & ~1.5-2 slower \\
UTT Nullifier \cite{tomescu2022utt} & \ding{55} & \ding{51} & \ding{51} & \ding{51} & Sigma+Pairing & 1.5x slower \\
\textbf{Our CRBN} & \ding{51} & \ding{51} & \ding{51} & \ding{51} & Sigma only & 1x (baseline) \\
\bottomrule
\end{tabular}
\end{center}
\vspace{1em}
\footnotesize{$^1$Operates on inputs (e.g., secret keys, attributes) hidden in commitments.} \\
\footnotesize{$^2$Nullifier can be presented as a commitment for unlinkability.} \\
\footnotesize{$^3$Ensures a unique underlying nullifier value per user-context pair for sybil resistance.} \\
\footnotesize{$^4$Approximate verification time relative to CRBN, based on benchmarks in Section X.}
\end{table}

Previous systems have addressed aspects of hierarchical credential binding and sybil resistance, but all have significant limitations:

\begin{itemize}
    \item \textbf{UTT}~\cite{tomescu2022utt} uses a similar approach to ours for anonymous payments, creating serial numbers (nullifiers) from a registration credential. However, UTT relies on bilinear pairings, introducing substantial computational overhead.
    
    \item \textbf{CanDID}~\cite{maram2021candid} clearly defines the master/context credential relationship but compromises privacy by maintaining mappings between credential public keys, enabling linkability across interactions.
    
    \item \textbf{Pairing-based systems} like SyRA~\cite{crites_syra_2024} and S3ID~\cite{rabaninejad_attribute-based_2024} implement hierarchical credentials but suffer from efficiency issues due to their reliance on pairings and S3ID uses Groth Sahai proofs which are less efficient than $\Sigma$-protocols.
    
    \item \textbf{Standard VRFs}~\cite{hutchison_verifiable_2005} could generate unique nullifiers but use pairings and reveal the user's public key during verification, compromising anonymity.
\end{itemize}

While our $\MIMCABC$ system (Chapter 3) provides efficient identity binding across multiple credentials, it doesn't address the hierarchical relationship between credentials or prevent a user from obtaining multiple credentials for the same context. This chapter fills that gap.

\subsection{Contributions}

We advance anonymous credential systems with three key contributions:

\begin{enumerate}
    \item \textbf{Pairing-Free VRF Construction}: We redesign the Dodis-Yampolskiy VRF for standard prime-order groups, eliminating pairings while maintaining security under the $q$-DDHI assumption. Our construction transforms $y = e(g, \tilde{g})^{1/(sk+x)}$ to $y = g^{1/(sk+x)}$ with a novel verification approach. We show a 3.1x speedup of our construction against the DY VRF. 
    
    \item \textbf{Zero-Knowledge Protocols for Inverse Relations}: We develop efficient $\Sigma$-protocols to prove knowledge of committed inverse exponents—a crucial building block for private nullifier verification. These protocols enable statements of the form:
    \[
    \zkpok\{(sk, x, r_1, r_2) : \cm_1 = g_1^{sk}g^{r_1} \land \cm_2 = g_2^{x}g^{r_2} \land y = g^{1/(sk+x)}\}
    \]
    
    \item \textbf{Credential Relationship Binding Nullifier (CRBN)}: We integrate these components into a complete system that binds master credentials to context credentials via nullifiers. Our two constructions, 1) producing a deterministic nullifier and 2) producing a committed nullifier, are respectively 5.5x and 2.9x faster for  $\VRFEval+\VRFProve$ and 4.6x and 3.9x faster for $\VRFVerify$ compared to the fastest pairing-based alternative we found \cite{tomescu2022utt}.
\end{enumerate}

These contributions enable a privacy-preserving credential hierarchy with sybil resistance, addressing a critical gap in existing anonymous credential systems. Our approach maintains the privacy guarantees of $\MIMCABC$ while adding the ability to enforce context-specific uniqueness—balancing privacy with the accountability requirements of real-world identity systems.






\section{Preliminaries}

\subsection{Cryptographic Assumptions}

\begin{definition}[q-DDHI Assumption]
Let $\mathbb{G}$ be a cyclic group of prime order $p$ with generator $g$. The $q$-Decisional-Diffie-Hellman Inversion ($q$-DDHI) \cite{mitsunari_new_2002} assumption states that for any PPT adversary $\mathcal{A}$, there exists a negligible function $\negl$ such that:
\[
\left|\Pr\left[ x \sample \Zp^*, b \sample \{0,1\}, z_0 = g^{1/x}, z_1 \sample \G : \mathcal{A}(g, g^x, g^{x^2}, \ldots, g^{x^q}, z_b) = b \right] - \frac{1}{2}\right| \leq \negl(\lambda)
\]
Informally, given $(g, g^x, g^{x^2}, \ldots, g^{x^q})$, no PPT adversary can distinguish $g^{1/x}$ from a random group element with non-negligible advantage.
\end{definition}

\begin{remark}
The $q$-DDHI assumption is a variant of the $(q+1)$-generalized Diffie-Hellman assumption as shown by Boneh and Boyen \cite{kanade_efficient_2004}. This assumption directly underpins the security of our pairing-free VRF construction.
\end{remark}

\begin{definition}[$q$-DBDHI Assumption]
Let $\mathbb{G}_1, \mathbb{G}_2, \mathbb{G}_T$ be cyclic groups of prime order $p$ with a bilinear pairing $e: \mathbb{G}_1 \times \mathbb{G}_2 \to \mathbb{G}_T$, and generators $g \in \mathbb{G}_1$, $\tilde{g} \in \mathbb{G}_2$. The $q$-Decisional-Bilinear Diffie-Hellman Inversion ($q$-DBDHI) assumption states that for any PPT adversary $\mathcal{A}$, there exists a negligible function $\negl$ such that:
\[
\left| \Pr\left[ x \sample \Z_p^*, b \sample \{0,1\}, z_0 = e(g, \tilde{g})^{1/x}, z_1 \sample \G_T : \mathcal{A}(g, g^x, g^{x^2}, \ldots, g^{x^q}, \tilde{g}, z_b) = b \right] - \frac{1}{2}\right| \leq \negl(\lambda)
\]
Informally, no PPT adversary can distinguish $e(g, \tilde{g})^{1/x}$ from a random element in $\mathbb{G}_T$ given $(g, g^x, \ldots, g^{x^q}, \tilde{g})$ with non-negligible advantage.
\end{definition}


\subsection{Building Blocks}

\subsubsection{Pedersen Vector Commitments}
We use position-binding Pedersen Commitments from Chapter 2, which allow committing to a vector of messages while hiding the values. For a message vector $[\id, \ctx, \exp, \k]$ and randomness $\usk$, the commitment is:
\[
\cm = \CMCom([m_1, \ldots, m_n];\usk) = g_1^{m_1} \cdots g_n^{m_n} g^\usk
\]

Pedersen Commitments provide three key properties:
\begin{itemize}
    \item \textbf{Hiding}: The commitment reveals no information about the committed values.
    \item \textbf{Binding}: It's computationally infeasible to open a commitment to different values.
    \item \textbf{Position-Binding}: Each position in the vector is cryptographically bound to its specific base element, preventing attribute swapping.
\end{itemize}


\subsubsection{Verifiable Random Functions}
A Verifiable Random Function (VRF) \cite{micali_verifiable_1999, hutchison_verifiable_2005} is a pseudorandom function that provides proofs of correct evaluation. Following \cite{bitansky_verifiable_2020}, a VRF consists of these algorithms:

\begin{definition}[Verifiable Random Function]
A VRF is a tuple of PPT algorithms \\
$(\VRFGen, \VRFEval, \VRFProve, \VRFVerify)$ with message space $\mathcal{X}$, output space $\mathcal{Y}$ and proof space $\Pi$ where:
\begin{itemize}
    \item $\VRFGen(1^\lambda) \to (sk, pk)$: Generates a secret key $sk$ and public key $pk$.
    \item $\VRFEval(sk,x) \to y$: Computes the VRF output $y \in \mathcal{Y}$ for input $x \in \mathcal{X}$ using secret key $sk$.
    \item $\VRFProve(sk,x) \to \pi$: Produces a proof $\pi \in \Pi$ that $y = \VRFEval(sk,x)$ is computed correctly.
    \item $\VRFVerify(pk,x,y,\pi) \to \{0,1\}$: Verifies that $y$ is the correct VRF output for input $x$ using proof $\pi$.
\end{itemize}
\end{definition}

A secure VRF must satisfy the following properties:

\begin{itemize}
    \item \textbf{Completeness:} Honest evaluation and proof generation always passes verification:
    \[
    \Pr\left[ \VRFVerify(pk,x,y,\pi) = 1 \ \middle| \ 
    \begin{array}{l}
        (sk, pk) \leftarrow \VRFGen(1^\lambda) \\
        y = \VRFEval(sk,x) \\
        \pi \leftarrow \VRFProve(sk,x)
    \end{array}
    \right] = 1
    \]
    
    \item \textbf{Uniqueness:} For each input $x$ and public key $pk$, only one output $y$ can be verified:
    \[
    \text{if} \quad \VRFVerify(pk, x, y_0, \pi_0) = \VRFVerify(pk, x, y_1, \pi_1) = 1 \quad \text{then} \quad y_0 = y_1
    \]
    
    \item \textbf{Pseudorandomness:} The VRF output is indistinguishable from random for any input not previously queried, defined by the following game $\mathcal{G}_{\mathcal{A}}^{\text{vrf}}$:
    \begin{enumerate}
        \item The VRF challenger samples $(sk, pk) \leftarrow \VRFGen(1^\lambda)$, and sends $pk$ to $\mathcal{A}$.
        \item $\mathcal{A}$ submits evaluation queries $x_1, \ldots, x_Q \in \mathcal{X}$, and receives $(y_i, \pi_i)$ for each query, where $y_i = \VRFEval(sk, x_i)$ and $\pi_i \leftarrow \VRFProve(sk, x_i)$.
        \item At any point, $\mathcal{A}$ submits a challenge input $x_* \in \mathcal{X}$ such that $x_* \not\in \{x_1, \ldots, x_Q\}$.
        \item The challenger computes $y_0^* = \VRFEval(sk, x_*)$, samples $y_1^* \sample \mathcal{Y}$ uniformly at random, then samples $b \sample \{0,1\}$ and sends $y_b^*$ to $\mathcal{A}$.
        \item $\mathcal{A}$ may continue to make evaluation queries for inputs other than $x_*$.
        \item At the end, $\mathcal{A}$ outputs a guess $b'$. The game outputs 1 if $b' = b$, and 0 otherwise.
    \end{enumerate}
    
    We say that the VRF satisfies pseudorandomness if for all PPT adversaries $\mathcal{A}$:
    \[
    \text{Adv}_{\mathcal{A}}^{\text{vrf}} := \left|\Pr\left[\mathcal{G}_{\mathcal{A}}^{\text{vrf}}(\lambda) = 1\right] - \frac{1}{2}\right| \leq \text{negl}(\lambda)
    \]
\end{itemize}


    
In our work, we focus on adapting the Dodis-Yampolskiy VRF \cite{hutchison_verifiable_2005}, which computes $y = e(g, \tilde{g})^{1/(sk+x)}$ in bilinear groups, to work efficiently in standard prime-order groups without pairings.

\subsubsection{Zero-Knowledge Proofs and Sigma-Protocols}
Our credential binding mechanism relies on zero-knowledge proofs, particularly Sigma-protocols, to verify relations between committed values without revealing them.

A Sigma-protocol is a three-move interactive proof system where:
\begin{enumerate}
    \item The prover $\mathcal{P}$ sends a commitment message $a$.
    \item The verifier $\mathcal{V}$ sends a random challenge $e$.
    \item The prover responds with $z$, and $\mathcal{V}$ accepts if the verification equation holds.
\end{enumerate}
These protocols satisfy:
\begin{itemize}
    \item \textbf{Completeness}: For all $(x,w) \in \mathcal{R}$, an honest prover always convinces the verifier.
    \item \textbf{Special Soundness}: There exists an efficient extractor $\mathcal{E}$ such that, given any statement $x$ and two accepting transcripts $(a,e,z)$ and $(a,e',z')$ with $e \neq e'$, $\mathcal{E}$ can extract a witness $w$ such that $(x,w) \in \mathcal{R}$.
    \item \textbf{Special Honest-Verifier Zero-Knowledge}: There exists an efficient simulator $\mathcal{S}$ that, given a statement $x$ and a challenge $e$, produces a transcript $(a,e,z)$ that is computationally indistinguishable from a real transcript between an honest prover and verifier, without using a witness.
\end{itemize}





\section{Technical Challenges in Credential Relationship Binding}

The three core challenges we solve are 1) remove pairings from DY's VRF. 2) Preserve Privacy. 3) Enable Unlinkability.


To understand our contributions, we examine the core technical challenges in creating an efficient, privacy-preserving nullifier mechanism for hierarchical credential systems. Consider a user with a master credential (containing secret key $\k$, denoted as $sk$ below) who needs to prove possession of a context credential (with identifier $\ctx = \text{"DMV"}$, denoted as $x$) without revealing their identity or allowing them to obtain multiple DMV credentials.

\subsection{Challenge 1: Creating a Pairing-Free Nullifier}

The Dodis-Yampolskiy \cite{hutchison_verifiable_2005} VRF provides an elegant solution for generating unique, pseudorandom nullifiers using $y = e(g, \tilde{g})^{1/(sk+x)}$ in bilinear groups. We wish to improve on the efficiency and privacy. 

\subsubsection{Can we remove Bilinear Pairings from DY VRF?}

We start with the challenge of removing pairings from the DY VRF. DY VRF achieves verifiability:

\begin{itemize}
    \item The verifier confirms $\pi = \tilde{g}^{1/(sk+x)}$ is valid via $e(g^x \cdot pk, \pi) = e(g^{sk+x}, \tilde{g}^{1/(sk+x)}) = e(g, \tilde{g})$
    \item Then checks $y = e(g, \pi) = e(g, \tilde{g})^{1/(sk+x)}$
\end{itemize}

Verification depends on the bilinearity property $e(g^a, \tilde{g}^b)^c = e(g, \tilde{g})^{abc}$, which enables "exponent multiplication" across groups, allowing the transformations
\[
e(g^x \cdot pk, \pi) \quad \stackrel{?}{=} \quad e(g^{sk+x}, \tilde{g}^{1/(sk+x)}) \quad \stackrel{?}{=} \quad 
 e(g, \tilde{g})^{(sk+x)/(sk+x)} \quad  \stackrel{}{=} \quad e(g, \tilde{g})
\]
\[
y \quad \stackrel{?}{=} \quad e(g, \pi) \quad \stackrel{?}{=} \quad e(g,\tilde{g}^{1/(sk+x)}) \quad \stackrel{}{=} \quad e(g, \tilde{g})^{1/(sk+x)}
\]

Security relies on the $q$-DBDHI problem which states that given $(g, g^x, g^{x^2}, \ldots, g^{x^q}, \tilde{g})$, no $\PPT$ adversary can distinguish between $e(g,\tilde{g})^{1/x}$ and a uniform element in $\G_T$ with non-negligible advantage , ensuring the VRF outputs maintain pseudorandomness after the adversary has observed $(x',y',\pi')$ pairs.

\subsubsection{The Prime-Order Group Challenge}

Prime-order groups without pairings don't have the bilinearity property, given $pk = g^{sk}$, $x$, and $y = g^{1/(sk+x)}$, standard group operations cannot directly verify the relationship. For example, given $pk  \cdot g^x = g^{sk+x}$, attempts fail to equate or cancel out:

\begin{enumerate}
    \item $g^{sk+x} \cdot g^{1/(sk+x)} = g^{sk+x+1/(sk+x)}$
    \item $g^{sk+x}/g^{1/(sk+x)} = g^{(sk+x)^2-1/(sk+x)}$
\end{enumerate}

Our insight is to reverse the verification approach. Instead of trying to derive $g^{1/(sk+x)}$ from $g^{sk+x}$ or cancel with a reciprocal, we use a zero knowledge proof $\Sigma$-protocol to verify that $y$ raised to the power $(sk+x)$ equals $g$. In doing so, our pairing-free construction shifts from the $q$-DBDHI assumption to the $q$-DDHI assumption. This gives us the relation:

\[
\mathcal{R}_{\mathsf{DY-PF}} = \left\{ 
\begin{array}{l} 
(\pk, x, y),\\
(sk) 
\end{array}
\ \middle|
\ \begin{array}{l}
pk = g^{sk} \\
y^{sk + x} = g  \\
\end{array} \right\}
\]

\subsection{Challenge 2: Privacy-Preserving Credential Binding}

Even with a pairing-free solution, standard VRFs reveal the public key $pk = g^{sk}$ and the input $x$,  compromising credential anonymity. In our hierarchical credential system, $sk$ and $x$ are committed attributes in different credentials that must remain private. We solve this through a two-step approach: first, addressing the simpler case of a single committed exponent, then extending to linear combinations.

\subsubsection*{Step 1. Proving Knowledge of a Committed Inverse}
Our first insight is reducing this problem to the problem of proving knowledge of a (committed) inverse exponent which is congruent to proving a DDHI challenge $g^{1/x}$
\[
\mathcal{R}_{\mathsf{DDHI}} = \left\{ 
\begin{array}{l} (\cm, y),\\
(x, r) 
\end{array}
\ \middle|
\ \begin{array}{l}
\cm = g_1^{x} g^r \\
y = g^{1/x} \\
\end{array} \right\}
\]

We use a $\Sigma$-protocol to prove knowledge of the exponents $x, r$; during the protocol, we use the same random coin and response from $x$ to show the relation $y^x (=g)$ holds, and therefore $y$ must commit to $x$. We show the full protocol \ref{pok-committed-inverse-exponent}. 

\subsubsection*{Step 2. Proving Knowledge of a Committed Inverse Linear Relation}

We then extend this to prove a linear relation of inverse exponents $y = g^{1/(sk+x)}$ where $sk$ and $x$ come from different commitments, enforcing a relation $\mathcal{R}_{\mathsf{nullifier}}$ holds. We show the full protocol here \ref{pok-committed-inverse-linear-relation}. 

\[
\mathcal{R}_{\mathsf{nullifier}} = \left\{ 
\begin{array}{l} (\cm_1, \cm_2, y),\\
(sk, x, r_1, r_2) 
\end{array}
\ \middle|
\ \begin{array}{l}
\cm_1 = g_1^{sk} g^{r_1} \\
\cm_2 = g_2^{x} g^{r_2} \\
y = g^{1/(sk + x)} \\
\end{array} \right\}
\]


\subsection{Challenge 3: Unlinkable Credential Presentations}

A deterministic nullifier $y = g^{1/(sk+x)}$ is necessary for Sybil resistance ensuring a unique, verifiable value for each $(sk,x)$ pair. However, deterministic algorithms create privacy problems with linkability. There are many use-cases for the verifiable connection of $(sk, x)$, such as revocation accumulators, which a user would need to interact with frequently, requiring a probabilistic output that can verify the $(sk+x)$ connection. 

Our solution is to commit to the nullifier rather than revealing it directly:
\[
\cm_{\mathsf{null}} = g_3^{1/(sk+x)} g^{r_3}
\]

We must now prove that the committed value represents the nullifiers $(sk+x)$, we do so with the following relation:


\[
\mathcal{R}_{\mathsf{CommittedNull}} = \left\{ 
\begin{array}{l} 
(\cm_1, \cm_2, \cm_3, \cm_4), \\
(sk, x, r_1, r_2, r_3, r_4) 
\end{array}
\ \middle| \
\begin{array}{l}
\cm_1 = g_1^{sk} g^{r_1} \\
\cm_2 = g_2^{x} g^{r_2} \\
\cm_3 = g_3^{1/(sk + x)} g^{r_3}  \quad \text{\small{The Committed Nullifier}}\\
\cm_4 = \cm_3^{sk + x} g^{r_4} = g_3 g^{r_3 (sk + x) + r_4}
\end{array} \right\}
\]

The auxiliary commitment $\cm_4$ enables verification analogous to our approach in Challenge 1: instead of verifying $y^{sk+x} = g$ directly, we prove that $\cm_3^{sk+x} = g_3$ (modulo some randomness $g^{r_4}$). This algebraic relationship can only hold if $\cm_3$ commits to $g_3^{1/(sk+x)}$, thus verifying the correctness of the committed nullifier without revealing it.

This construction extends our technique from Challenge 2 by applying the exponentiation-inversion verification paradigm entirely within the commitment space. The verification operates on the same mathematical principle ($x \cdot \frac{1}{x} = 1$), but now uses homomorphic properties of Pedersen commitments to prove the relationship between committed values, enabling both unlinkable presentations and verifiable credential binding.




\subsection{Our Approach}

To address these challenges, we develop a series of progressively more sophisticated cryptographic constructions:

\begin{enumerate}
    \item \textbf{Pairing-Free VRF (DY-PF):} We design a $\Sigma$-protocol that enables verification of $y = g^{1/(sk+x)}$ without pairings by proving the relation $y^{sk+x} = g$, preserving the security properties of the original DY VRF but with significantly improved efficiency.
    
    \item \textbf{Privacy-Preserving VRF:} We extend our technique to verify nullifiers computed from committed attributes, creating novel $\Sigma$-protocols for proving inverse exponent relations between commitments.
    
    \item \textbf{Committed Nullifier:} We further develop our framework to commit the nullifier itself, enabling unlinkable credential presentations while maintaining Sybil resistance.
\end{enumerate}

Each construction builds upon the previous one, culminating in the Credential Relationship Binding Nullifier (CRBN) that satisfies all our requirements for hierarchical, privacy-preserving, and efficient credential systems.































































\newpage
\section{Introduction OLD}






\newpage
\section{Dodis-Yampolskiy VRF: Core Structure and Security Properties}

The Dodis-Yampolskiy VRF~\cite{hutchison_verifiable_2005} operates in a bilinear group setting with prime-order groups $\mathbb{G}_1$, $\mathbb{G}_2$, and $\mathbb{G}_T$, with a Type-3 pairing $e: \mathbb{G}_1 \times \mathbb{G}_2 \rightarrow \mathbb{G}_T$. Let $g \in \mathbb{G}_1$ and $\tilde{g} \in \mathbb{G}_2$ be generators. The construction is as follows:

\begin{itemize}
    \item $\mathsf{VRF.Gen}(1^\lambda) \to (sk, pk)$: Sample $sk \sample \mathbb{Z}_p^*$, compute $pk = g^{sk} \in \mathbb{G}_1$. Output $(sk, pk)$.
    
    \item $\mathsf{VRF.Eval}(sk, x) \to y$: Compute $y = e(g, \tilde{g})^{1/(sk + x)} \in \mathbb{G}_T$.
    
    \item $\mathsf{VRF.Prove}(sk, x) \to \pi$: Compute proof $\pi = \tilde{g}^{1/(sk + x)} \in \mathbb{G}_2$.
    
    \item $\mathsf{VRF.Vfy}(pk, x, y, \pi) \to \{0, 1\}$: Verify two equations:
    \begin{align}
        e(g^{x} \cdot pk, \pi) &\stackrel{?}{=} e(g, \tilde{g}) \quad \text{(Equation 1)}\\
        y &\stackrel{?}{=} e(g, \pi) \quad \text{(Equation 2)}
    \end{align}
\end{itemize}

The security of this VRF rests upon the $q$-Decisional Bilinear Diffie-Hellman Inversion ($q$-DBDHI) assumption, which states that given $(g, g^{sk}, g^{(sk)^2}, \ldots, g^{(sk)^q})$ and $\tilde{g}$, the value $e(g, \tilde{g})^{1/(sk+x)}$ is computationally indistinguishable from a random element in $\mathbb{G}_T$.

\subsubsection{Analysis of Security Properties}

To understand the security of the Dodis-Yampolskiy VRF, we analyze how its construction achieves the three essential VRF properties:

\subsubsection*{Uniqueness}
Information-theoretic uniqueness is enforced by the algebraic properties of the pairing. 
Verification Equation 1 enforces that only one valid proof $\pi$ can exist for each input, which in turn determines a unique output $y$ through Equation 2. The first equation verifies the algebraic relationship $y^{sk+x} = e(g, \tilde{g})$ without revealing $sk$:
\begin{align}
    e(g^{x} \cdot pk, \pi) &= e(g^{x} \cdot g^{sk}, \tilde{g}^{1/(sk + x)}) \\
    &= e(g^{sk + x}, \tilde{g}^{1/(sk + x)}) \\
    &= e(g, \tilde{g})^{(sk + x) \cdot 1/(sk + x)} \\
    &= e(g, \tilde{g})
\end{align}

This uniqueness guarantee is crucial: for each $(sk,x)$ pair, only one value $\pi = \tilde{g}^{1/(sk+x)}$ can satisfy Equation 1. Consequently, only one output value $y = e(g, \pi) = e(g, \tilde{g})^{1/(sk+x)}$ can satisfy Equation 2.


\subsubsection*{Pseudorandomness}
The VRF output $ y = e(g, \tilde{g})^{1/(sk + x)} \in \mathbb{G}_T $ is pseudorandom under the $ $-DBDHI assumption, which implies that $ e(g, \tilde{g})^{1/(sk + x)} $ is computationally indistinguishable from a random element in $ \mathbb{G}_T $ given $ (g, g^{sk}, \ldots, g^{(sk)^q}) $ and $ \tilde{g} $. Unlike a Verifiable Unpredictable Function (VUF) which has weaker security assumptions allowing some structural relationship leakage for repeated queries, the DY VRF places $ y $ in $ \mathbb{G}_T $ and the proof $ \pi = \tilde{g}^{1/(sk + x)} $ in $ \mathbb{G}_2 $. This separation, combined with the one-way bilinear pairing $ e(g, \pi) = y $, ensures outputs appear fully random and independent across inputs, achieving pseudorandomness rather than mere unpredictability.


\subsubsection*{Correctness} holds with honest protocol run





\section{Step 1. Pairing-Free VRF in Prime-Order Groups (DY-PF)}\label{sec-dy-pf}

In this section, we present our pairing-free verifiable random function (VRF), P-DY, which operates in a prime-order group and eliminates the computational overhead of bilinear pairings found in the Dodis-Yampolskiy (DY) VRF. We detail the construction, including a $\Sigma$-protocol for proof generation, and demonstrate that it satisfies correctness, uniqueness, and pseudorandomness. Notably, P-DY trades the information-theoretic uniqueness of DY for computational uniqueness, a shift we justify through the security analysis.

\subsection{Construction}

Our VRF operates in a prime-order group $\mathbb{G}$ of order $p$ with generator $g$. The message space is $\mathcal{X} = \mathbb{Z}_p$, the output space is $\mathcal{Y} = \mathbb{G}$, and the proof space is $\Pi = \mathbb{G} \times \mathbb{G} \times \mathbb{Z}_p$. The algorithms are defined as follows:

\begin{itemize}
    \item $\mathsf{VRF.Gen}(1^\lambda) \to (sk, pk)$: Sample $sk \sample \mathbb{Z}_p^*$, compute $pk = g^{sk}$, and output $(sk, pk)$.
    \item $\mathsf{VRF.Eval}(sk, x) \to y$: Compute $y = g^{1/(sk + x)} \in \mathbb{G}$.
    \item $\mathsf{VRF.Prove}(sk, x) \to \pi$: Generate proof $\pi$ using the $\Sigma$-protocol described below.
    \item $\mathsf{VRF.Verify}(pk, x, y, \pi) \to \{0, 1\}$: Output 1 if $\pi$ verifies $y$ correctly per the $\Sigma$-protocol, else 0.
\end{itemize}

\begin{remark}
    Verifying $\VRFVerify(pk, x, \VRFEval(sk, x) \to y) \to 1$ is a naive verification approach without a proof which yeilds a Verifiable Unpredictable Function (VUF), not a VRF because it lacks the mechanism to prove pseudorandomness to a verifier. DY uses pairings to bridge the gap, we replace pairings with a $\Sigma$-protocol. 
\end{remark}

\subsection{Proof Protocol}
\begin{protocol}{P-DY Proof Protocol}{}\label{protocol-pdy-protocol1}
\textbf{Common Input:} $g, pk, y \in \mathbb{G}$, $x \in \mathbb{Z}_p$ \\
\textbf{Prover Input:} $sk \in \mathbb{Z}_p^*$ with $pk = g^{sk}$, $y = g^{1/(sk + x)}$ \\
\textbf{Relation: }
\[
\mathcal{R} = \left\{ (\pk, x, y), (sk) \ \middle| pk = g^{sk} \land y^{sk + x} = g \right\}
\]
\begin{enumerate}
    \item \textbf{Commitment:} Prover samples $r \sample  \mathbb{Z}_p$, computes $T_1 = g^r$, $T_2 = y^r$, sends $(T_1, T_2)$.
    \item \textbf{Challenge:} Verifier samples $c \sample  \mathbb{Z}_p$, sends $c$.
    \item \textbf{Response:} Prover computes $z = r + c \cdot (sk + x)$, sends $z$.
    \item \textbf{Verification:} Verifier checks: $g^z \stackrel{?}{=} T_1 \cdot (pk \cdot g^x)^c$ and $y^z \stackrel{?}{=} T_2 \cdot g^c$
\end{enumerate}
\end{protocol}

\subsection{Security Analysis}

\begin{remark}
P-DY trades the \emph{information-theoretic uniqueness} of the DY VRF, enforced by pairings, for \emph{computational uniqueness} via the soundness of the $\Sigma$-protocol and the discrete logarithm assumption. This eliminates pairings while maintaining security against polynomial-time adversaries.
\end{remark}

\subsubsection{Correctness}

Correctness requires that an honest prover’s output $y$ and proof $\pi$ always pass verification. For $pk = g^{sk}$, $y = g^{1/(sk + x)}$, $T_1 = g^r$, $T_2 = y^r$, and $z = r + c(sk + x)$, the verification equations hold:
\begin{align*}
g^z &= g^{r + c(sk + x)} = g^r \cdot g^{c(sk + x)} = g^r \cdot (g^{sk} \cdot g^x)^c = T_1 \cdot (pk \cdot g^x)^c \\
y^z &= y^{r + c(sk + x)} = y^r \cdot y^{c(sk + x)} = y^r \cdot (y^{sk + x})^c = y^r \cdot g^c = T_2 \cdot g^c
\end{align*}

Since $y^{sk + x} = g^{1/(sk + x) \cdot (sk + x)} = g$, both checks pass, confirming correctness.

\subsubsection{Uniqueness}

Uniqueness ensures that, for a fixed $pk$ and $x$, only one $y$ can be successfully verified. In DY, pairings enforce this information theoretically. In P-DY, uniqueness is computational, relying on the discrete logarithm problem.

For a valid $y$, $y^{sk + x} = g$, so $y = g^{1/(sk + x)}$ is unique in $\mathbb{G}$. Suppose an adversary produces $y' \neq y$ with a valid proof $\pi'$. Then $y'^{sk + x} = g$ and $y^{sk + x} = g$, implying $(y'/y)^{sk + x} = 1$. In a prime-order group, $y'/y = g^k$ for some $k \neq 0$, so $y' = y \cdot g^k$. But $y'^{sk + x} = (y \cdot g^k)^{sk + x} = g \cdot g^{k(sk + x)} = g$ requires $g^{k(sk + x)} = 1$, which holds only if $k(sk + x) = 0 \pmod{p}$. For random $sk$ and $x$, $sk + x = 0$ is negligible. Alternatively, if $y'$ corresponds to a different $sk'$ where $pk = g^{sk'}$, finding $sk' \neq sk$ breaks the discrete logarithm assumption.

Thus, producing a distinct verifiable $y'$ is computationally infeasible, ensuring uniqueness.

\subsubsection{Pseudorandomness}

Pseudorandomness requires that $y = g^{1/(sk + x)}$ appears random in $\mathbb{G}$ without knowledge of $sk$, even given other input-output pairs. We rely on the $q$-DDHI assumption, which states that $g^{1/(sk + x)}$ is indistinguishable from random given $(g, g^{sk}, \ldots, g^{(sk)^q})$ for polynomial $q$. 
The $\Sigma$-protocol is zero-knowledge, leaking no information about $sk$ beyond $pk$.

\begin{proof}[Sketch]
    Assume an adversary can distinguish $y$ from random, solving $q$-DDHI. The challenger simulates proofs for $q$ inputs using $g^{sk^i}$ for challenge $x^*$, provides $y^* = g^{a/(sk + x^*)}$ or a random element. A successful distinguished implies a $q$-DDHI solver which is assumed to be a hard problem.
\end{proof}





\newpage
\section{Step 2: Privacy-Preserving VRF}

The P-DY VRF, relies on a public key $pk = g^{sk}$ for verification. This exposes $pk$, potentially compromising user privacy in applications like anonymous credential systems. To preserve privacy, we use committed attributes and zero-knowledge proofs, proving the VRF output $y = g^{1/(sk + x)}$ is correctly computed without revealing $sk$ or $x$. We build this up by first solving for the $q$-DDHI challenge before applying the linear relation proof for additive exponents. 

\subsection{Step 2.1: Proving Committed Exponent and DDHI Challenge}

We begin with a simplified relation: given a commitment $\cm = g_1^x g^r$ and an element $y = g^{1/x}$, the prover demonstrates knowledge of $x$ and $r$ satisfying both equations, without revealing them. This is linked to the Decisional Diffie-Hellman Inversion (DDHI) assumption—where distinguishing $g^{1/x}$ from a random element given $g, g^x$ is computationally hard—making it a secure and essential starting point for our privacy-preserving construction.

\begin{protocol}{Proving Knowledge of Inverse Exponent}{committed-inverse-exponent}\label{pok-committed-inverse-exponent}
\textbf{Common Input:} Group generators $g_1, g \in \mathbb{G}$, commitment $\cm \in \mathbb{G}$, and element $y \in \mathbb{G}$ \\
\textbf{Prover Input:} Witness $(x, r)$ such that $\cm = g_1^x g^r$ and $y = g^{1/x}$ \\
\textbf{Relation:} 
\[
\mathcal{R} = \left\{ (\cm, y), (x, r) \ \middle|\ \cm = g_1^x g^r \ \land\ y = g^{1/x} \right\}
\]
\begin{enumerate}
    \item \textbf{Commitment:} Prover samples $a_x, a_r \sample \mathbb{Z}_q$ and computes:
    \[
    T_1 = g_1^{a_x} g^{a_r}, \quad T_y = y^{a_x}
    \]
    Sends $(T_1, T_y)$ to the verifier.

    \item \textbf{Challenge:} Verifier samples $c  \sample  \mathbb{Z}_q$ and sends $c$ to the prover.

    \item \textbf{Response:} Prover computes:
    \[
    z_x = a_x + c \cdot x, \quad z_r = a_r + c \cdot r
    \]
    Sends $(z_x, z_r)$ to the verifier.

    \item \textbf{Verification:} Verifier checks:
    \[
    T_1 \cdot \cm^c \stackrel{?}{=} g_1^{z_x} g^{z_r}, \quad T_y \cdot g^c \stackrel{?}{=} y^{z_x}
    \]
\end{enumerate}
\end{protocol}

This protocol achieves two goals:
\begin{enumerate}
    \item Commitment Opening: The equation $T_1 \cdot \cm^c = g_1^{z_x} g^{z_r}$ proves the prover knows $x$, $r$ satisfying the commitment opening.
    \item The equation $T_y \cdot g^c = y^{z_x}$ holds because $y^{a_x} \cdot g^c = y^{a_x} \cdot y^{c \cdot x} = y^{a_x + c \cdot x} = y^{z_x}$ if $y^x = g$, proving $y = g^{1/x}$ links to the same $x$ in $\cm$.
\end{enumerate}






\newpage
\subsection{Step 2.1: Proving Committed Exponents with Linear Relation}

Having established a zero-knowledge proof for a committed exponent tied to an inverse, we now extend this to handle two committed values, $sk$ and $x$, such that $y = g^{1/(sk + x)}$. This step introduces a linear relation between $sk$ and $x$, which is essential for the VRF output. The prover demonstrates knowledge of $sk$, $x$, and their respective randomness values, proving the relation without revealing the values themselves.


\begin{protocol}{Proving Committed Inverse Linear Relation}{}\label{pok-committed-inverse-linear-relation}
\textbf{Common Input:} Group generators $g_1, g_2, g \in \mathbb{G}$, $\cm_1, \cm_2, y \in \mathbb{G}$ \\
\textbf{Prover Input:} Witness $(sk, x, \usk_1, \usk_2)$ such that $\cm_1 = g_1^{sk} g^{\usk_1}$, $\cm_2 = g_2^{x}g^{\usk_2}$ and $ y = g^{1/(sk + x)}$ \\
\textbf{Relation: }
\[
\mathcal{R} = \{(\cm_1,\cm_2, y), (sk, x, \usk_1, \usk_2) \mid \cm_1 = g_1^{sk} g^{\usk} \wedge \cm_2 = g_2^{x}g^{\usk_2} \wedge y = g^{1/(sk + x)}\}
\]
\begin{enumerate}
    \item \textbf{Commitment:} Prover computes:
    \begin{align*}
        a_{sk}, a_x, a_{r_1}, a_{r_2} &\sample \Z_q & T_1 &\gets g_1^{a_{sk}} g^{a_{r_1}} & T_2 &\gets g_2^{a_x} g^{a_{r_2}} & T_y &\gets y^{a_{sk} + a_x}
    \end{align*}
    Sends $(T_1, T_2, T_y)$ to verifier.
    
    \item \textbf{Challenge:} Verifier samples $c \sample \mathbb{Z}_q$ and sends to prover.
    
    \item \textbf{Response:} Prover computes:
     \begin{align*}
        z_{sk} &= a_{sk} + c \cdot sk & z_x &= a_x + c \cdot x &  z_m &= (a_{sk} + a_x) + c \cdot (sk + x)\\   
        z_{r_1} &= a_{r_1} + c \cdot r_1 & z_{r_2} &= a_{r_2} + c \cdot r_2
    \end{align*}
    Sends $(z_{sk}, z_x, z_{r_1} z_{r_2}, z_m)$ to verifier.
    
    \item \textbf{Verification:} Verifier checks:
    \begin{align*}
        T_1 \cdot \cm_1^c &\stackrel{?}{=} g_1^{z_{sk}}g^{z_{r_1}} 
        &
        T_2 \cdot \cm_2^c &\stackrel{?}{=}  g_2^{z_x} g^{z_{r_2}} 
        &
        T_y \cdot g^c &\stackrel{?}{=} y^{z_m} &
        z_m &\stackrel{?}{=} z_{sk} + z_x
    \end{align*}
\end{enumerate}
\end{protocol}







\newpage
\section{Privacy-Preserving Committed Nullifier for Hierarchical Credentials}
\label{sec:committed-nullifier}

In the Multi-Issuer Multi-Credential Attribute-Based Anonymous Credential (MIMC-ABC) system, nullifiers ensure sybil resistance by enforcing uniqueness of context-specific credentials bound to a master credential. However, exposing nullifiers as public values, as in our pairing-free Verifiable Random Function (VRF) construction (Section~\ref{sec-dy-pf}), risks linkability across presentations, compromising user privacy. To address this, we propose a novel zero-knowledge protocol that commits the nullifier within a Pedersen commitment, proving its correctness relative to committed attributes from master and context credentials without revealing sensitive information.

Our goal is to design a Sigma-protocol that proves a commitment $\cm_3 = g_3^{1/(sk + x)} g^{r_3}$ contains the inverse exponent $1/(sk + x)$, where $sk$ and $x$ are attributes committed in $\cm_1 = g_1^{sk} g^{r_1}$ (master credential) and $\cm_2 = g_2^{x} g^{r_2}$ (context credential), respectively. This enables privacy-preserving sybil resistance, allowing nullifiers to be used in applications like set membership proofs without linking user actions. The protocol operates in prime-order groups, avoiding the computational overhead of bilinear pairings, and integrates seamlessly with MIMC-ABC's efficient Sigma-protocol framework.

The novelty of our approach lies in its pairing-free, zero-knowledge proof of a committed inverse linear relation, a significant advancement over prior VRF-based nullifiers that rely on pairings or expose inputs \cite{hutchison_verifiable_2005,tomescu2022utt}. By committing the nullifier, we achieve unlinkability, a critical requirement for anonymous credential systems in regulatory contexts like KYC/AML. Furthermore, our protocol is 33\% faster in evaluation and 60\% faster in verification compared to pairing-based schemes (Section~\ref{sec:performance}), offering practical scalability for real-world deployments.

This contribution is crucial for enabling hierarchical, sybil-resistant anonymous credentials that balance privacy, accountability, and efficiency. It addresses practical challenges in federated identity systems, where users must prove credential relationships without compromising anonymity, and lays the foundation for advanced functionalities like private revocation and accumulator-based proofs.

\begin{protocol}{Proving Committed Nullifier for VRF}{committed-nullifier-vrf}\label{pok-committed-nullifier-vrf}
\textbf{Common Input:} Group generators $g_1, g_2, g_3, g \in \mathbb{G}$, commitments $\cm_1, \cm_2, \cm_3, \cm_4 \in \mathbb{G}$ \\
\textbf{Prover Input:} Witness $(sk, x, r_1, r_2, r_3, r_4)$ such that:
    \[
    \cm_1 = g_1^{sk} g^{r_1}, \quad \cm_2 = g_2^{x} g^{r_2}, \quad \cm_3 = g_3^{1/(sk + x)} g^{r_3}, \quad \cm_4 = \cm_3^{sk + x} g^{r_4} = g_3 g^{r_3 (sk + x) + r_4}
    \]
\textbf{Relation:}
\[
\mathcal{R} = \left\{ 
\begin{array}{l} 
(\cm_1, \cm_2, \cm_3, \cm_4), \\
(sk, x, r_1, r_2, r_3, r_4) 
\end{array}
\ \middle| \
\begin{array}{l}
\cm_1 = g_1^{sk} g^{r_1} \\
\cm_2 = g_2^{x} g^{r_2} \\
\cm_3 = g_3^{1/(sk + x)} g^{r_3} \\
\cm_4 = \cm_3^{sk + x} g^{r_4} = g_3 g^{r_3 (sk + x) + r_4}
\end{array} \right\}
\]
\begin{enumerate}
    \item \textbf{Commitment:} Prover samples $a_{sk}, a_x, a_{r_1}, a_{r_2}, a_{r_3}, a_{r_4} \sample \mathbb{Z}_q$ and computes:
       \[
       T_1 = g_1^{a_{sk}} g^{a_{r_1}}, \quad T_2 = g_2^{a_x} g^{a_{r_2}}, \quad T_3 = g_3^{a_{\beta}} g^{a_{r_3}}, \quad T_4 = \cm_3^{a_{sk} + a_x} g^{a_{r_4}}
       \]
       where $a_{\beta} = 1/(a_{sk} + a_x)$. Sends $(T_1, T_2, T_3, T_4)$ to verifier.
    
    \item \textbf{Challenge:} Verifier samples $c \sample \mathbb{Z}_q$ and sends to prover.
    
    \item \textbf{Response:} Prover computes:
       \[
       z_{sk} = a_{sk} + c \cdot sk, \quad z_x = a_x + c \cdot x, \quad z_{r_1} = a_{r_1} + c \cdot r_1, \quad z_{r_2} = a_{r_2} + c \cdot r_2
       \]
       \[
       z_{r_3} = a_{r_3} + c \cdot r_3, \quad z_{r_4} = a_{r_4} + c \cdot r_4
       \]
       Sends $(z_{sk}, z_x, z_{r_1}, z_{r_2}, z_{r_3}, z_{r_4})$ to verifier.
    
    \item \textbf{Verification:} Verifier checks:
       \[
       T_1 \cdot \cm_1^c \stackrel{?}{=} g_1^{z_{sk}} g^{z_{r_1}}, \quad T_2 \cdot \cm_2^c \stackrel{?}{=} g_2^{z_x} g^{z_{r_2}}, \quad T_3 \cdot \cm_3^c \stackrel{?}{=} g_3^{z_{\beta}} g^{z_{r_3}}
       \]
       \[
       T_4 \cdot \cm_4^c \stackrel{?}{=} \cm_3^{z_{sk} + z_x} g^{z_{r_4}}
       % \cm_4 \stackrel{?}{=} g_3 g^{s} \text{ for some } s \in \mathbb{Z}_q
       \]
\end{enumerate}
\end{protocol}






















































% \section{Step 2. Privacy}
% Our VRF builds upon the $q$-DDHI assumption and this is where we start our zero knowledge proof protocol. Specifically, we want to prove (in zero knowledge) that we know $\alpha$ such that $g^{1/\alpha}$ 

% Let $G$ be a cyclic group of prime order $q$, with generators $g_1, g \in G$. Assume the discrete logarithm problem (DLP) is hard in $G$, and the discrete log of $g_1$ with respect to $g$ is unknown.
% The prover aims to prove in zero-knowledge that there exists $m \in \mathbb{Z}_q^*$ such that $\alpha \cdot \beta = 1 \mod q$, where $\beta = \frac{1}{\alpha}$, given commitments to $\alpha$ and $\beta$.

% Challenges
% Sigma protocols prove knowledge of exponents, proving knowledge of an inverse exponent. 

% INSERT PROOF PROTOCOL FOR DDHI Challenge

% \subsubsection*{Completeness}:
% We show that if the prover follows the protocol honestly, all verification equations will be satisfied.

% \begin{enumerate}
%     \item The first two verification equations ($T_1 \cdot \cm_1^c \stackrel{?}{=} g_1^{z_\alpha} g^{z_{\usk_1}}$, $T_2 \cdot \cm_2^c \stackrel{?}{=} g_1^{z_\beta} g^{z_{\usk_2}}$) are simple Schnorr proofs of exponents which we will not expand 
        
%     \item Third verification equation: $T_3 \cdot \cm_3^c \stackrel{?}{=} \cm_2^{z_\alpha} g^{z_{\usk_3}}$
%     \begin{align*}
%         T_3 \cdot \cm_3^c &= \cm_2^{a_\alpha} g^{a_{\usk_3}} \cdot (\cm_2^\alpha g^{\usk_3})^c \\
%         &= \cm_2^{a_\alpha} g^{a_{\usk_3}} \cdot \cm_2^{\alpha \cdot c} g^{\usk_3 \cdot c} \\
%         &= \cm_2^{a_\alpha + \alpha \cdot c} g^{a_{\usk_3} + \usk_3 \cdot c} \\
%         &= \cm_2^{z_\alpha} g^{z_{\usk_3}}
%     \end{align*}
    
%     \item Fourth verification equation: $T_4 \cdot \cm_4^c \stackrel{?}{=} g^{z_{\usk_4}}$
%     \begin{align*}
%         T_4 \cdot \cm_4^c &= g^{a_{\usk_4}} \cdot (g^{\usk_4})^c \\
%         &= g^{a_{\usk_4}} \cdot g^{\usk_4 \cdot c} \\
%         &= g^{a_{\usk_4} + \usk_4 \cdot c} \\
%         &= g^{z_{\usk_4}}
%     \end{align*}
    
%     \item Fifth verification equation: $\frac{\cm_3}{\cm_4} \stackrel{?}{=} g_1$
    
%     Using the relations $\beta = \frac{1}{\alpha}$ and $\usk_4 = \usk_3 + \alpha \cdot \usk_2$:
%     \begin{align*}
%         \cm_3 &= \cm_2^\alpha g^{\usk_3} \\
%         &= (g_1^\beta g^{\usk_2})^\alpha g^{\usk_3} \\
%         &= g_1^{\beta \cdot \alpha} g^{\usk_2 \cdot \alpha} g^{\usk_3} \\
%         &= g_1^{\frac{1}{\alpha} \cdot \alpha} g^{\usk_2 \cdot \alpha + \usk_3} \\
%         &= g_1 g^{\usk_3 + \alpha \cdot \usk_2} \\
%         &= g_1 g^{\usk_4} \\
%         &= g_1 \cdot \cm_4
%     \end{align*}
    
%     Therefore:
%     \begin{align*}
%         \frac{\cm_3}{\cm_4} = \frac{g_1 \cdot \cm_4}{\cm_4} = g_1
%     \end{align*}
% \end{enumerate}

% Since all verification equations are satisfied when the prover follows the protocol, we conclude that our sigma protocol satisfies the completeness property.


% \subsubsection*{Soundness}
% We prove the special soundness property by demonstrating that given two accepting transcripts with identical first messages but different challenges, we can extract a valid witness satisfying all relations required by the protocol.

% Let $(T_1, T_2, T_3, T_4, c, z_\alpha, z_\beta, z_{\usk_1}, z_{\usk_2}, z_{\usk_3}, z_{\usk_4})$ and $(T_1, T_2, T_3, T_4, c', z'_\alpha, z'_\beta, z'_{\usk_1}, z'_{\usk_2}, z'_{\usk_3}, z'_{\usk_4})$ be two accepting transcripts with $c \neq c'$.

% \begin{enumerate}
%     \item \textbf{Witness extraction:}
    
%     Following the standard extraction technique for sigma protocols, we compute each witness component:
%     \begin{align}
%     \alpha &= \frac{z_\alpha - z'_\alpha}{c - c'} & x &= \frac{z_\beta - z'_\beta}{c - c'} \\
%     \usk_1 &= \frac{z_{\usk_1} - z'_{\usk_1}}{c - c'} & \usk_2 &= \frac{z_{\usk_2} - z'_{\usk_2}}{c - c'} \\
%     \usk_3 &= \frac{z_{\usk_3} - z'_{\usk_3}}{c - c'} & \usk_4 &= \frac{z_{\usk_4} - z'_{\usk_4}}{c - c'}
%     \end{align}
    
%     These extracted values are guaranteed to be consistent with the protocol's verification equations precisely because both transcripts are accepting.
    
%     \item \textbf{Verification of commitment relations:}
    
%     We first show that the extracted witness satisfies $\cm_1 = g_1^m g^{\usk_1}$. From the acceptance of both transcripts, we have:
%     \begin{align}
%         T_1 \cdot \cm_1^c &= g_1^{z_\alpha} g^{z_{\usk_1}} \\
%         T_1 \cdot \cm_1^{c'} &= g_1^{z'_\alpha} g^{z'_{\usk_1}}
%     \end{align}
    
%     Dividing these equations and substituting our extracted values:
%     \begin{align}
%         \cm_1^{c-c'} &= g_1^{z_\alpha-z'_\alpha} g^{z_{\usk_1}-z'_{\usk_1}} \\
%         &= g_1^{\alpha(c-c')} g^{\usk_1(c-c')} \\
%         \Rightarrow \cm_1 &= g_1^\alpha g^{\usk_1}
%     \end{align}
    
%     Using identical extraction and verification techniques with the corresponding verification equations, we establish that:
%     \begin{align}
%         \cm_2 &= g_1^\beta g^{\usk_2} \\
%         \cm_3 &= \cm_2^\alpha g^{\usk_3} \\
%         \cm_4 &= g^{\usk_4}
%     \end{align}
    
%     \item \textbf{Verification of the multiplicative inverse relation ($\beta = \frac{1}{\alpha}$):}
    
%     The core cryptographic property enforced by our protocol is the multiplicative inverse relationship between $m$ and $x$. This is where the protocol's security guarantee is ultimately derived from. We verify this relationship using the final verification equation $\frac{\cm_3}{\cm_4} = g_1$.
    
%     Substituting our extracted witness values:
%     \begin{align}
%         \frac{\cm_3}{\cm_4} &= g_1 \\
%         \frac{\cm_2^\alpha g^{\usk_3}}{g^{\usk_4}} &= g_1 \\
%         \cm_2^\alpha g^{\usk_3 - \usk_4} &= g_1
%     \end{align}
    
%     Further substituting $\cm_2 = g_1^x g^{\usk_2}$:
%     \begin{align}
%         (g_1^x g^{\usk_2})^\alpha g^{\usk_3 - \usk_4} &= g_1 \\
%         g_1^{x \cdot \alpha} g^{\alpha \cdot \usk_2 + \usk_3 - \usk_4} &= g_1
%     \end{align}
    
%     In the generic group model, this equation can only hold if:
%     \begin{align}
%         \beta \cdot \alpha &= 1 \\
%         \alpha \cdot \usk_2 + \usk_3 - \usk_4 &= 0
%     \end{align}
    
%     These equations immediately give us the two critical relations:
%     \begin{align}
%         \beta &= \frac{1}{\alpha} \\
%         \usk_4 &= \usk_3 + \alpha \usk_2
%     \end{align}
    
%     The first relation confirms that our protocol successfully enforces the multiplicative inverse relationship between $\alpha$ and $\beta$. The second relation verifies the consistency of the randomizers across the commitments, ensuring that no malicious prover can construct valid-looking commitments without knowing values that satisfy all required relations.
% \end{enumerate}

% We have thus demonstrated that our extractor obtains a complete and valid witness $(\alpha, \beta, \usk_1, \usk_2, \usk_3, \usk_4)$ satisfying all relations in the statement, establishing the special soundness property of our sigma protocol. This ensures that no prover can successfully convince a verifier without knowledge of values satisfying the multiplicative inverse relationship $\beta = \frac{1}{\alpha}$ and the associated consistency requirements.

% \subsubsection*{Zero-Knowledge}
% We prove the honest-verifier zero-knowledge property by presenting a simulator that produces transcripts indistinguishable from real protocol executions without knowing any witness.

% \begin{enumerate}
%     \item \textbf{Simulator Construction:}
    
%     Given public input $(g_1, g, \cm_1, \cm_2, \cm_3, \cm_4)$ and any challenge $c \in \mathbb{Z}_q$, our simulator:
    
%     \begin{itemize}
%         \item Samples $z_\alpha, z_\beta, z_{\usk_1}, z_{\usk_2}, z_{\usk_3}, z_{\usk_4} \sample \mathbb{Z}_q$ uniformly at random
        
%         \item Computes the first message by working backwards from the verification equations:
%         \begin{align}
%             T_1 &\gets g_1^{z_\alpha} g^{z_{\usk_1}} \cdot \cm_1^{-c} \\
%             T_2 &\gets g_1^{z_\beta} g^{z_{\usk_2}} \cdot \cm_2^{-c} \\
%             T_3 &\gets \cm_2^{z_\alpha} g^{z_{\usk_3}} \cdot \cm_3^{-c} \\
%             T_4 &\gets g^{z_{\usk_4}} \cdot \cm_4^{-c}
%         \end{align}
        
%         \item Outputs the transcript $(T_1, T_2, T_3, T_4, c, z_\alpha, z_\beta, z_{\usk_1}, z_{\usk_2}, z_{\usk_3}, z_{\usk_4})$
%     \end{itemize}
    
%     \item \textbf{Perfect Indistinguishability:}
    
%     In a real protocol execution with an honest prover, the responses have the form $z_\alpha = a_\alpha + c \cdot \alpha$ where $a_\alpha$ are uniformly random values. Since $a_\alpha \sample \mathbb{Z}_q$ and addition with $c \cdot \alpha$ is a permutation over $\mathbb{Z}_q$, the distribution of real responses is uniform over $\mathbb{Z}_q$.
    
%     The simulator directly samples responses uniformly from $\mathbb{Z}_q$, matching this distribution exactly. Given these responses and the challenge $c$, the first message components $T_i$ are uniquely determined by the verification equations in both real and simulated cases.
    
%     Consequently, the joint distribution of $(T_1, T_2, T_3, T_4, c, z_\alpha, z_\beta, z_{\usk_1}, z_{\usk_2}, z_{\usk_3}, z_{\usk_4})$ is identical for both the simulator and the honest protocol.
% \end{enumerate}

% This construction demonstrates that our protocol satisfies perfect honest-verifier zero-knowledge. Notably, the simulator succeeds without requiring knowledge of the critical multiplicative inverse relationship $\beta = \frac{1}{\alpha}$ or any other witness values, confirming that the protocol reveals no information about these secret values beyond what is already implied by the statement being proven.






















\subsection{Performance Analysis}
Benchmarks were performed on MacBook Air M2 16GB RAM using our Rust implementation with the arkworks library \cite{arkworks_contributors_arkworks_2022}. Compared to the state-of-the-art nullifier scheme from \cite{tomescu2022utt},




\begin{table}[ht]
\begin{center}
\caption{Verification Time for 4 Credentials with Varying Attributes (time in ms)}
\label{tab:performance-vrf}
\begin{tabular}{l@{\hspace{1em}}r@{\hspace{2em}}r@{\hspace{5em}}r@{\hspace{2em}}r}
\toprule
\textbf{Scheme} & \multicolumn{2}{c}{\textbf{Eval + Prove (ms)}} & \multicolumn{2}{c}{\textbf{Verify (ms)}} \\
\cmidrule(lr){2-3} \cmidrule(lr){4-5}
& \textbf{ms} & \textbf{Speedup} & \textbf{ms} & \textbf{Speedup} \\
\midrule
DY$^1$ \cite{hutchison_verifiable_2005}                     & 1.27 &        & 2.2   &       \\
Our DY-PF \ref{sec-dy-pf}                                   & 0.41 & 3.1x   & 0.70  & 3.1x  \\
\midrule
DY Private \cite{tomescu2022utt}                            & 5.91 &        & 6.47  &       \\
Our DY-PF-Private \ref{pok-committed-inverse-linear-relation}                      & 1.08 & 5.5x   & 1.41  & 4.6x  \\
Our DY-PF-Private-ComOut \ref{pok-committed-nullifier-vrf}       & 2.01 & 2.9x   & 1.65  & 3.9x  \\
\bottomrule
\end{tabular}
\par\medskip
\raggedright
\footnotesize{$^1$We use optimized pairing for verification by computing all pairings in Miller Loop format before a single Final Exponentiation, reducing verify time from 2.85(ms) to 2.27(ms), a 1.26x speedup.}

\footnotesize{$^2$[Your second footnote text here]} 
\end{center}
\end{table}



